\documentclass[]{book}
\usepackage{lmodern}
\usepackage{amssymb,amsmath}
\usepackage{ifxetex,ifluatex}
\usepackage{fixltx2e} % provides \textsubscript
\ifnum 0\ifxetex 1\fi\ifluatex 1\fi=0 % if pdftex
  \usepackage[T1]{fontenc}
  \usepackage[utf8]{inputenc}
\else % if luatex or xelatex
  \ifxetex
    \usepackage{mathspec}
  \else
    \usepackage{fontspec}
  \fi
  \defaultfontfeatures{Ligatures=TeX,Scale=MatchLowercase}
\fi
% use upquote if available, for straight quotes in verbatim environments
\IfFileExists{upquote.sty}{\usepackage{upquote}}{}
% use microtype if available
\IfFileExists{microtype.sty}{%
\usepackage{microtype}
\UseMicrotypeSet[protrusion]{basicmath} % disable protrusion for tt fonts
}{}
\usepackage[margin=1in]{geometry}
\usepackage{hyperref}
\hypersetup{unicode=true,
            pdftitle={431 Project Instructions},
            pdfauthor={Thomas E. Love},
            pdfborder={0 0 0},
            breaklinks=true}
\urlstyle{same}  % don't use monospace font for urls
\usepackage{natbib}
\bibliographystyle{apalike}
\usepackage{longtable,booktabs}
\usepackage{graphicx,grffile}
\makeatletter
\def\maxwidth{\ifdim\Gin@nat@width>\linewidth\linewidth\else\Gin@nat@width\fi}
\def\maxheight{\ifdim\Gin@nat@height>\textheight\textheight\else\Gin@nat@height\fi}
\makeatother
% Scale images if necessary, so that they will not overflow the page
% margins by default, and it is still possible to overwrite the defaults
% using explicit options in \includegraphics[width, height, ...]{}
\setkeys{Gin}{width=\maxwidth,height=\maxheight,keepaspectratio}
\IfFileExists{parskip.sty}{%
\usepackage{parskip}
}{% else
\setlength{\parindent}{0pt}
\setlength{\parskip}{6pt plus 2pt minus 1pt}
}
\setlength{\emergencystretch}{3em}  % prevent overfull lines
\providecommand{\tightlist}{%
  \setlength{\itemsep}{0pt}\setlength{\parskip}{0pt}}
\setcounter{secnumdepth}{5}
% Redefines (sub)paragraphs to behave more like sections
\ifx\paragraph\undefined\else
\let\oldparagraph\paragraph
\renewcommand{\paragraph}[1]{\oldparagraph{#1}\mbox{}}
\fi
\ifx\subparagraph\undefined\else
\let\oldsubparagraph\subparagraph
\renewcommand{\subparagraph}[1]{\oldsubparagraph{#1}\mbox{}}
\fi

%%% Use protect on footnotes to avoid problems with footnotes in titles
\let\rmarkdownfootnote\footnote%
\def\footnote{\protect\rmarkdownfootnote}

%%% Change title format to be more compact
\usepackage{titling}

% Create subtitle command for use in maketitle
\newcommand{\subtitle}[1]{
  \posttitle{
    \begin{center}\large#1\end{center}
    }
}

\setlength{\droptitle}{-2em}

  \title{431 Project Instructions}
    \pretitle{\vspace{\droptitle}\centering\huge}
  \posttitle{\par}
    \author{Thomas E. Love}
    \preauthor{\centering\large\emph}
  \postauthor{\par}
      \predate{\centering\large\emph}
  \postdate{\par}
    \date{Version: 2018-09-11 10:57:44}

\usepackage{booktabs}
\usepackage{amsthm}
\makeatletter
\def\thm@space@setup{%
  \thm@preskip=8pt plus 2pt minus 4pt
  \thm@postskip=\thm@preskip
}
\makeatother

\usepackage{amsthm}
\newtheorem{theorem}{Theorem}[chapter]
\newtheorem{lemma}{Lemma}[chapter]
\theoremstyle{definition}
\newtheorem{definition}{Definition}[chapter]
\newtheorem{corollary}{Corollary}[chapter]
\newtheorem{proposition}{Proposition}[chapter]
\theoremstyle{definition}
\newtheorem{example}{Example}[chapter]
\theoremstyle{definition}
\newtheorem{exercise}{Exercise}[chapter]
\theoremstyle{remark}
\newtheorem*{remark}{Remark}
\newtheorem*{solution}{Solution}
\begin{document}
\maketitle

{
\setcounter{tocdepth}{1}
\tableofcontents
}
\hypertarget{overview}{%
\chapter*{Overview}\label{overview}}
\addcontentsline{toc}{chapter}{Overview}

This website contains the Fall 2018 project information for PQHS / CRSP
/ MPHP 431: Statistical Methods in Biological \& Medical Sciences,
Section 1.

\begin{itemize}
\tightlist
\item
  All materials related to the project (including these instructions)
  are maintained and linked at
  \url{https://github.com/THOMASELOVE/431-2018-project}.
\item
  The direct link to this document is
  \url{https://thomaselove.github.io/431-2018-project}.
\end{itemize}

\hypertarget{your-project-includes-two-studies}{%
\section*{Your Project includes Two
Studies}\label{your-project-includes-two-studies}}
\addcontentsline{toc}{section}{Your Project includes Two Studies}

Your final project for this course will result in a portfolio of work
related to two studies.

\textbf{Study 1 - Class Survey}. In the first study, you (sometimes
working individually, sometimes in a group) will design, administer,
analyze and present the results of a survey designed to compare two or
three groups of subjects on some \emph{categorical} and
\emph{quantitative} outcomes we will develop from your initial ideas.

\textbf{Study 2 - Your Data}. In the second study, you (working
individually) will propose a research question and relevant data of
interest to you, and then complete all elements of a data science
project designed to create a statistical model for a \emph{quantitative}
outcome, then use it for prediction and assess the quality of those
predictions.

\hypertarget{you-have-nine-tasks-to-complete-this-semester}{%
\section*{You have Nine Tasks to Complete this
Semester}\label{you-have-nine-tasks-to-complete-this-semester}}
\addcontentsline{toc}{section}{You have Nine Tasks to Complete this
Semester}

The project involves two analyses (one for the class survey and one for
your personal study), and a total of 9 tasks (deliverables.) Each task
is to be completed by \textbf{12 NOON} on the specified date.

\begin{itemize}
\tightlist
\item
  \protect\hyperlink{taskA}{Task A (The Proposal)} is due at noon on
  2018-10-12
\item
  \protect\hyperlink{taskB}{Task B (Presentation Sign-Up)} is
  \textbf{also} due at noon on 2018-10-12
\item
  \protect\hyperlink{taskC}{Task C (Survey Editing)} involves group work
  and is due at noon on 2018-10-23
\item
  \protect\hyperlink{taskD}{Task D (Survey Comparison Plan)} is
  \textbf{also} due at noon on 2018-10-23
\item
  \protect\hyperlink{taskE}{Task E (Taking the Survey)} is due at noon
  on 2018-10-31
\item
  \protect\hyperlink{taskF}{Task F (Sharing Study 2 Data)} is due at
  noon on 2018-11-14
\item
  \protect\hyperlink{taskG}{Task G (The Update)} is due at noon on
  2018-11-28
\item
  \protect\hyperlink{taskH}{Task H (The Portfolio)} is due at noon on
  2018-12-13
\item
  \protect\hyperlink{taskI}{Task I (Your Presentation)} will be held on
  2018-12-10, 2018-12-11 or 2018-12-13
\end{itemize}

The bulk of this document contains specific instructions for each of
these tasks.

\hypertarget{working-with-this-document}{%
\section*{Working with This Document}\label{working-with-this-document}}
\addcontentsline{toc}{section}{Working with This Document}

\begin{enumerate}
\def\labelenumi{\arabic{enumi}.}
\tightlist
\item
  This document is broken down into multiple sections. Use the table of
  contents at left to navigate.
\item
  At the top of the document, you'll see icons which you can click to

  \begin{itemize}
  \tightlist
  \item
    search the document,
  \item
    change the size, font or color scheme of the page, and
  \item
    download a PDF or EPUB (Kindle-readable) version of the entire
    document.
  \end{itemize}
\item
  The document is a work in progress, and will be updated occasionally
  through the semester. Check the Version information above to verify
  the last update time\footnote{Note that the ePub and PDF versions will
    show slightly different times (but on the same day) as the HTML
    version.}.
\end{enumerate}

\hypertarget{need-help}{%
\section*{Need Help?}\label{need-help}}
\addcontentsline{toc}{section}{Need Help?}

Questions about the project or the course can be directed to
\textbf{431-help at case dot edu} or to Dr.~Love directly at
\texttt{thomas\ dot\ love\ at\ case\ dot\ edu}.

\begin{itemize}
\tightlist
\item
  The course home page is at
  \url{https://github.com/THOMASELOVE/431-2018}
\end{itemize}

\hypertarget{project-objectives}{%
\chapter{Project Objectives}\label{project-objectives}}

It is hard to learn statistics (or anything else) passively; concurrent
theory and application are
essential\footnote{Though by no means an original idea, this particular phrasing is stolen from Harry Roberts.}

\hypertarget{study-1-is-about-making-comparisons-and-visualizing-groups-of-data.}{%
\section{Study 1 is about making comparisons and visualizing groups of
data.}\label{study-1-is-about-making-comparisons-and-visualizing-groups-of-data.}}

\textbf{Study 1} involves data from a \textbf{class survey}, to be
conducted in October. We will design, administer, analyze and present
survey results designed to compare two or three groups of subjects from
the class on some \emph{categorical} and \emph{quantitative} outcomes.
In the analysis stage, everyone will be working with different parts of
the same data set.

\begin{quote}
Think of a graph as a comparison. All graphs are comparisons (indeed,
all statistical analyses are comparisons). If you already have the graph
in mind, think of what comparisons it's enabling. Or if you haven't
settled on the graph yet, think of what comparisons you'd like to make.
\href{http://andrewgelman.com/2014/03/25/statistical-graphics-course-statistical-graphics-advice/}{Andrew
Gelman}
\end{quote}

In your eventual analysis of Study 1, you will be comparing both
quantitative and categorical outcomes across 2-3 groups. All tools
necessary for Study 1 are in Parts A and B of the course, and include
the following\ldots{}

\begin{itemize}
\tightlist
\item
  Descriptive and exploratory summaries of the data across the groups
  for each of your chosen outcomes, including, of course, attractive and
  well-constructed visualizations, graphs and tables.
\item
  Comparisons of the population mean difference for at least one
  quantitative outcome across a set of two (or three) groups, including
  appropriate demonstrations of the reasons behind the choices you made
  between parametric, non-parametric and bootstrap procedures.
\item
  Comparisons of the population proportions for at least one categorical
  outcome across your set of two (or three) groups, including
  appropriately interpreted point estimates and confidence intervals.
\end{itemize}

Note well that Study 1 is \textbf{not} about building sophisticated
statistical models, and using them to make predictions. That's Study 2.

\hypertarget{study-2-is-about-building-a-model-and-making-predictions.}{%
\section{Study 2 is about building a model, and making
predictions.}\label{study-2-is-about-building-a-model-and-making-predictions.}}

\textbf{Study 2} involves data about a \textbf{research question that
you will propose}, involving data of interest to you. Thus, everyone
will be working with a different data set. You will complete all
elements of a data science project designed to create a statistical
model for a \emph{quantitative} outcome, then use it for prediction, and
assess the quality of those predictions.

\begin{quote}
All models are wrong but some are useful.
\href{https://en.wikipedia.org/wiki/All_models_are_wrong}{George E. P.
Box}
\end{quote}

In Study 2, you will be building a multiple linear regression model, and
using it to predict a quantitative outcome of interest. The tools
necessary for Study 2 appear in each Part of the course, especially Part
C, and include the following\ldots{}

\begin{itemize}
\tightlist
\item
  Describing the experimental or observational study design used to
  gather the data, as well as the complete data collection process.
\item
  Sharing the complete raw data in an appropriate way with a
  statistician (Dr.~Love). This means that, in general, data including
  protected health information are \emph{not} appropriate for this
  project.
\item
  Developing appropriate research questions that lead to the
  identification of smart measures for predictors and outcomes, and then
  the development of a prediction model using multiple linear
  regression.
\item
  Using a training sample to develop a model, and present the process
  that leads to a final set of 2-3 candidate models in the training
  sample.
\item
  Using a test sample to evaluate the quality of predictions from each
  of the candidate models, and making a final selection.
\item
  Evaluating the adherence of the data you've collected to the
  assumptions of multiple linear regression, and iterating through the
  model-building process as necessary until the final model shows no
  strong violations of those assumptions.
\end{itemize}

\hypertarget{why-two-studies}{%
\section{Why Two Studies?}\label{why-two-studies}}

The main reason is that I can't figure out a way to get you to think
about all of the things I hope you'll learn from this project in a
single study.

\begin{enumerate}
\def\labelenumi{\arabic{enumi}.}
\tightlist
\item
  I set different tasks for Study 1 and for Study 2, allowing us to
  touch on a wider fraction of the things I hope you'll learn in 431.
\item
  I want some of the work to be done as a class, some in groups, some as
  individuals.
\item
  Some of you have easy access to great data you want to study in this
  class, and in fact, that's a primary motivation for taking the class.
  But not all of you.
\item
  I have to evaluate each of your projects, and there are many students
  in the class. Knowing at least one of the data sets you'll be working
  with helps me manage this.
\item
  Having a broad range of activities to evaluate helps reduce the cost
  of a mistake on any one of them, so that we can build on what you do
  well.
\item
  All of Study 1 can be done by the middle of November, leaving the last
  few weeks of the semester for you to focus on Study 2.
\end{enumerate}

\hypertarget{educational-objectives}{%
\section{Educational Objectives}\label{educational-objectives}}

\begin{quote}
``Statistics has no reason for existence except as the catalyst for
investigation and discovery.''
\href{https://en.wikipedia.org/wiki/George_E._P._Box}{George E. P. Box}
\end{quote}

I am primarily interested in your learning something interesting, useful
and even valuable from your project. An effective project will
demonstrate:

\begin{enumerate}
\def\labelenumi{\arabic{enumi}.}
\tightlist
\item
  The ability to create and formulate research questions that are
  statistically and scientifically appropriate.
\item
  The ability to turn research questions into measures of interest.
\item
  The ability to pull and merge and clean and tidy data, then present
  the data set following
  \href{https://github.com/jtleek/datasharing}{Jeff Leek's guide to
  sharing data with a statistician}.
\item
  The ability to identify appropriate estimation / testing procedures
  for the class survey using both continuous and categorical outcomes.
\item
  The ability to build a reasonable model, including interactions and
  transformations to deal with non-linearity, assess the quality of the
  model and residual plots, then use the model to make predictions.
\item
  The ability to build a Table 1 to showcase potential differences
  between variables.
\item
  The ability to identify and (with help) solve problems that crop up
\item
  The ability to comment on your work within code, and in written and
  oral presentation.
\item
  The ability to build a Markdown-based report and a Markdown-based set
  of slides for presentation.
\end{enumerate}

\hypertarget{taskA}{%
\chapter{Task A (The Proposal) Instructions}\label{taskA}}

Task A requires you to accomplish the following:

On 2018-09-24, you will become part of a \textbf{group} of about five
people, and your group will:

\begin{enumerate}
\def\labelenumi{\arabic{enumi}.}
\tightlist
\item
  develop and propose 2-3 ``research questions'' for Study 1 (The Class
  Survey). Here, your research questions must clearly identify
  meaningful statistical comparisons.
\item
  propose 6-10 ``homemade'' survey questions for Study 1 that relate to
  your research questions, and
\item
  propose a ``scale'' for Study 1 that also relates to your research
  questions.
\end{enumerate}

As an \textbf{individual}, you will also:

\begin{enumerate}
\def\labelenumi{\arabic{enumi}.}
\setcounter{enumi}{3}
\tightlist
\item
  develop and propose a meaningful research question for Study 2 (Your
  Data). This question needs to clearly relate to modeling and
  prediction of a quantitative outcome on the basis of a set of
  predictor variables.
\item
  identify and present a data set that is likely to lead to an answer to
  the research question proposed for Study 2, and that is appropriate
  for use in this project.
\end{enumerate}

The rest of this section contains guidance as to what sort of questions
your group will need to propose for the class survey in Study 1, and as
to what sorts of data sets and research questions are appropriate for
your Study 2 proposal.

\hypertarget{deadline-and-submission-information}{%
\section{Deadline and Submission
information}\label{deadline-and-submission-information}}

The project groups will only apply to Project Tasks A, C and D. You will
become part of a project group on 2018-09-24 and the groups will disband
after the survey is finalized in late October.

Task A is due at noon on 2018-10-12. You will submit your Task A work
through \href{https://canvas.case.edu/}{Canvas}.

\begin{itemize}
\tightlist
\item
  The group work (Parts 1-3) need to be submitted by one of the members
  of your group.

  \begin{itemize}
  \tightlist
  \item
    If you are not the person submitting this information for your
    group, then your Task A submission should begin with the statement:
    ``Parts 1-3 of Task A were submitted for my group by PERSON'S
    NAME.''
  \end{itemize}
\item
  All students need to submit Parts 4 and 5 of Task A, individually.
\item
  Please note that \textbf{Task A and Task B} are due at the same time.
\end{itemize}

\hypertarget{study-1-work-for-task-a}{%
\section{Study 1 work for Task A}\label{study-1-work-for-task-a}}

In Study 1, you will survey your fellow students through an online
instrument (containing somewhere around 100 items) that we will develop
in Tasks A, C and D in September and October and then administer in the
final week of October (Task E).

Students in the class will develop the items for this instrument in 10
groups of about 5 people per group. The final survey will include
questions generated by each of the 10 groups in the class.

The course survey will be done online, and the respondents
(de-identified, of course) will include all students in the current 431
class, plus the teaching assistants, and perhaps some volunteers from
previous iterations of the course, in an effort to get a reasonable (but
by no means random or representative) sample of graduate students at
CWRU.

Remember that Study 1 is about making comparisons between groups.

\hypertarget{research-questions}{%
\subsection{Research Questions}\label{research-questions}}

The first part of Task A requires your group to develop and propose 2-3
``research questions'' for Study 1 (The Class Survey).

\begin{quote}
A research question is the fundamental core of a research project,
study, or review of literature. It focuses the study, determines the
methodology, and guides all stages of inquiry, analysis, and reporting.
\href{https://researchrundowns.com/intro/writing-research-questions/}{Source}
\end{quote}

\begin{itemize}
\tightlist
\item
  The research questions your project group will prepare for Study 1
  should state the study objective in terms that allow us to apply
  statistical methods to test data to obtain an answer.
\item
  Each research question should be written in the form of a comparison
  of 2-3 exposures or groups in terms of an outcome.
\item
  At least one of your research questions needs to compare groups on a
  quantitative outcome, and at least one needs to compare groups on a
  categorical outcome (containing no more than 5 categories.)
\end{itemize}

Quoting Roger Peng, from
\href{https://bookdown.org/rdpeng/exdata/}{Exploratory Data Analysis
with R}:

\begin{quote}
Formulating a question can be a useful way to guide the exploratory data
analysis process and to limit the exponential number of paths that can
be taken with any sizeable dataset. In particular, a sharp question or
hypothesis can serve as a dimension reduction tool that can eliminate
variables that are not immediately relevant to the question. For
example, (suppose that we are interested in) looking at an air pollution
dataset from the U.S. Environmental Protection Agency (EPA).
\end{quote}

\begin{quote}
A general question one could ask is ``Are air pollution levels higher on
the east coast than on the west coast?'' But a more specific question
might be ``Are hourly ozone levels on average higher in New York City
than they are in Los Angeles?''
\end{quote}

\begin{quote}
Note that both questions may be of interest, and neither is right or
wrong. But the first question requires looking at all pollutants across
the entire east and west coasts, while the second question only requires
looking at single pollutant in two cities. It's usually a good idea to
spend a few minutes to figure out what is the question you're really
interested in, and narrow it down to be as specific as possible (without
becoming uninteresting).
\end{quote}

\hypertarget{checklist-for-research-questions}{%
\subsubsection{Checklist for Research
Questions}\label{checklist-for-research-questions}}

\begin{enumerate}
\def\labelenumi{\arabic{enumi}.}
\tightlist
\item
  Is our research question (RQ) something that we are curious about and
  that others might care about?
\item
  Does our RQ present an issue on which we can justify a stand prior to
  data collection about what we think will happen?
\item
  Is our RQ too broad, too narrow, or OK, given the time frame and
  restrictions of this survey?
\item
  Is our RQ measurable? What type of information do we need? Can I find
  a way to ask a limited number of survey questions in such a way to
  allow me to (after the data are collected) either support or
  contradict a position on my RQ?
\end{enumerate}

\begin{itemize}
\tightlist
\item
  Adapted from
  \href{http://www8.esc.edu/htmlpages/writerold/menus.htm\#develop}{this
  online tutorial from Empire State College}
\end{itemize}

\hypertarget{tips-on-writing-good-research-questions}{%
\subsubsection{Tips on Writing Good Research
Questions}\label{tips-on-writing-good-research-questions}}

\begin{itemize}
\tightlist
\item
  \href{http://twp.duke.edu/uploads/media_items/research-questions.original.pdf}{Duke}
  has a nice overview online of key issues.
\item
  \href{http://vanderbilt.edu/writing/manage/wp-content/uploads/2013/06/Formulating\%20Your\%20Research\%20Question.pdf}{Vanderbilt}
  has some nice materials, built from the
  \href{http://www8.esc.edu/htmlpages/writerold/menus.htm\#develop}{tutorial
  at Empire State College} quoted earlier
\item
  Jeff Leek provides several relevant tips in \emph{The Elements of Data
  Analytic Style}
\item
  \url{https://researchrundowns.com/intro/writing-research-questions/}
  has some excellent tips on wording
\end{itemize}

\hypertarget{specifying-survey-questions}{%
\subsection{Specifying Survey
Questions}\label{specifying-survey-questions}}

The second part of Task A requires your group to develop and propose
6-10 ``homemade'' survey questions for Study 1 that relate to your
research questions.

Your group will need to specify the exact wording for your survey
questions (and potential answers for any categorical responses.) This
will likely require some editing and rework, once we have the complete
set of proposed questions from all students. Be prepared to revise and
resubmit, quickly, so that all items can be resolved in time for
publication of the draft survey.

Of your 6-10 survey questions \ldots{}

\begin{itemize}
\tightlist
\item
  at least two should ask the respondents to provide you with a
  \textbf{number} that expresses a quantitative outcome of interest to
  you, and these outcomes should relate closely to at least one of your
  research questions.

  \begin{itemize}
  \tightlist
  \item
    If you are asking people to respond to a prompt using a rating, that
    rating should be expressed on a wide scale. Our preference is a
    0-100 scale for quantitative items, where 0 represents the most
    negative reaction and 100 the most positive reaction to the item.
  \item
    One common choice is to make a statement and ask for agreement with
    that statement on a scale from 0 = Strongly Disagree to 100 =
    Strongly Agree.
  \item
    The reason we prefer a 0-100 scale is to increase variation in our
    responses, as compared to, say, a 1-10 scale.
  \item
    When responding to items using a scale like this on the actual
    survey, please use the whole scale.
  \end{itemize}
\item
  \textbf{at least two} should ask the respondents to provide you with a
  response that expresses a categorical outcome of interest to you, and
  these, too, should relate to at least one of your research questions.

  \begin{itemize}
  \tightlist
  \item
    You will need to specify each of the available responses that you
    wish to use in the survey.
  \item
    No more than five options for your categorical outcome, please.
  \item
    The response options you specify should be mutually exclusive and
    collectively exhaustive.
  \end{itemize}
\item
  \textbf{at least two} should ask the respondents to categorize
  themselves into one of two (or three) groups.

  \begin{itemize}
  \tightlist
  \item
    Be aware that you will need to have at least 10 students in each
    group in order to build a semi-reasonable analysis.
  \item
    You should expect that 50-55 people will actually respond to the
    survey, in total.
  \item
    Again, these groupings should be linked to your research questions.
  \end{itemize}
\end{itemize}

You are welcome to submit exactly 6, or as many as 10 total survey
questions in this part of the Task. It is likely that some of your
survey questions will not correspond to some of your research questions,
and that's OK, but each survey question should be linked to at least one
of your research questions.

\hypertarget{dr.love-will-include-15-survey-questions-automatically}{%
\subsubsection{Dr.~Love will include 15 Survey Questions
Automatically}\label{dr.love-will-include-15-survey-questions-automatically}}

The following items will be included in the survey to identify
``groups'' of students in a reasonable way. As a result, you should not
ask these questions in your proposed list, although you can and should
consider whether these groupings would be good candidates for
application to your research questions.

The following 7 items will have yes/no responses, and thus produce
binary groups for analysis.

\begin{enumerate}
\def\labelenumi{\arabic{enumi}.}
\tightlist
\item
  Were you born in the United States?
\item
  Is English the language you speak better than any other?
\item
  Do you identify as female?
\item
  Do you wear prescription glasses or contact lenses?
\item
  Before taking 431, had you ever used R before?
\item
  Are you currently married or in a stable domestic relationship?
\item
  Have you smoked 100 cigarettes or more in your entire life?
\end{enumerate}

The next eight items generate non-binary responses. Together, after the
survey is complete, we will identify ``cutpoints'' for these eight items
to identify groups of meaningful size.

\begin{enumerate}
\def\labelenumi{\arabic{enumi}.}
\setcounter{enumi}{7}
\tightlist
\item
  In what year were you born?
\item
  How would you rate your current health overall (Excellent, Very Good,
  Good, Fair, Poor)
\item
  For how long, in months, have you lived in Northeast Ohio?
\item
  What is your height in inches? (If you are five feet, eight inches
  tall, please write 68 inches. To convert from centimeters to inches,
  multiply your height in centimeters by 0.3937, and then round the
  result to the nearest inch.)
\item
  What is your weight in pounds? (To convert from kilograms to pounds,
  multiply your weight in kilograms by 2.2046, and then round the result
  to the nearest pound.)
\item
  What is your pulse rate, in beats per minute? (Please either use a
  tracking device, or count your pulse for 15 seconds then multiply by
  4)
\item
  Last week, on how many days did you exercise? (0 - 7)
\item
  Last night, how many hours of sleep did you get?
\end{enumerate}

\hypertarget{permitted-types-of-items}{%
\subsubsection{Permitted Types of
Items}\label{permitted-types-of-items}}

The survey will be conducted using a Google Form, rather than Survey
Monkey or some other tool. Thus, we have a somewhat restricted set of
item types.

For \textbf{quantitative measures}, Google Forms permit the use of

\begin{enumerate}
\def\labelenumi{\arabic{enumi}.}
\tightlist
\item
  a \emph{short answer} item without any restrictions on the response
  (except a character limit)
\item
  a \emph{short answer} item where respondents are forced to insert a
  number within a given range through a validation process that only
  accepts the response if it falls within the specified limits.
\item
  \emph{linear scale} items for ordered categorical ratings (but only on
  a scale of up to ten points - i.e.~1 to 10)
\end{enumerate}

For \textbf{categorical measures}, Google Forms permit the use of

\begin{enumerate}
\def\labelenumi{\arabic{enumi}.}
\tightlist
\item
  \emph{multiple choice} items for endorsing a single choice from a
  group of 2-10 alternatives.
\item
  \emph{checkbox} items for the endorsement of one or more choices from
  a group of 2-10 alternatives.
\item
  \emph{linear scale} items for ordered categorical ratings (often on a
  1-X scale, where X is between 2 and 10)
\item
  \emph{dropdown} items for selections of one option from a group of 2-3
  choices.
\end{enumerate}

\hypertarget{old-class-surveys}{%
\subsubsection{Old Class Surveys}\label{old-class-surveys}}

The surveys from 2014, 2015, 2016 and 2017 (some of which were developed
under a different set of project rules and requirements) have been
provided to you as PDF documents on our web site.

\hypertarget{i-am-here}{%
\chapter{I AM HERE!}\label{i-am-here}}

\begin{itemize}
\tightlist
\item
  Old surveys (with more than 100 items each) are available for you to
  view. If you are curious about questions that have been used in the
  past, here are links to
  \href{https://docs.google.com/a/case.edu/forms/d/e/1FAIpQLScEgrawEzqKZNyYQZj3_AvqOzo8ay5NpqvEIg-Bwpcg2fqMKw/viewform?c=0\&w=1}{the
  2014 survey},
  \href{https://docs.google.com/a/case.edu/forms/d/e/1FAIpQLSduLMePv2ZGnjL8zM9a2p_H8JS1-ux0W1Xe3A6iDZW5MxJnMA/viewform}{the
  2015 survey},
  \href{https://github.com/THOMASELOVE/431project-2017/blob/master/TaskB/2016_431_class_survey.pdf}{the
  2016 survey} and
  \href{https://github.com/THOMASELOVE/431project-2017/blob/master/TaskC/2017-10-31_final_version_classprojectsurvey.pdf}{the
  2017 survey}.
\end{itemize}

\hypertarget{extra-task-for-pqhs-msphd-students}{%
\subsection{Extra task for PQHS MS/PhD
students}\label{extra-task-for-pqhs-msphd-students}}

Students in MS or PhD programs in the Department of Population and
Quantitative Health Sciences will also need to specify a published scale
(available for free public use) to generate an outcome or grouping(s) of
interest from those completing the survey.

A scale is a published list of items, usually accompanied by a scoring
rubric. Examples of scales we have used in the past include:

\begin{itemize}
\tightlist
\item
  Two \href{https://chirr.nlm.nih.gov/health-orientation.php}{Health
  Consciousness Scales}, one due to Gould another to Dutta-Bregman
  \href{https://chirr.nlm.nih.gov/health-orientation.php}{Gould Health
  Consciousness Scale}
\item
  The
  \href{https://gosling.psy.utexas.edu/scales-weve-developed/ten-item-personality-measure-tipi/ten-item-personality-inventory-tipi/}{Ten-Item
  Personality Inventory}
\item
  The
  \href{https://das.nh.gov/wellness/Docs/Percieved\%20Stress\%20Scale.pdf}{Perceived
  Stress Scale}
\item
  The
  \href{https://www.sleepapnea.org/assets/files/pdf/ESS\%20PDF\%201990-97.pdf}{Epworth
  Sleepiness Scale}
\end{itemize}

Those outside the EPBI MS/PhD track are permitted to submit a scale as
well, if they would prefer to use such a scale instead of some part of
their homemade group of 5 questions. Please indicate this preference
clearly in your proposal.

\hypertarget{study-2-work-for-task-a}{%
\section{Study 2 Work for Task A}\label{study-2-work-for-task-a}}

You will present a proposal \textbf{summary} (\textless{} 300 words) and
a brief \textbf{data description} for Study 2 in Task A.

You will be building a multiple regression model, and using it to
predict your outcome of interest.

\begin{itemize}
\tightlist
\item
  We prefer data sets for this work containing 250 to 250,000
  observations, including at least one quantitative outcome, and at
  least four predictor variables (one of which may be identified as the
  ``key'' predictor of interest.)
\item
  Predictors may be quantitative or categorical.
\item
  If you would like to use a data set which does not meet these
  specifications, contact Dr.~Love via email well in advance of the Task
  A deadline (2016-10-11) to explain your reasons so that he can either
  approve your choice of data set, or not, in time for you to find a new
  data set.
\end{itemize}

\hypertarget{the-proposal-summary}{%
\subsection{The Proposal Summary}\label{the-proposal-summary}}

Take the time to come up with a good, interesting title. You are going
to work hard on this thing; please resist the temptation to kill my
interest at the start by calling it ``EPBI 431 Statistics Project.''

Provide me a very brief summary of what you're trying to accomplish -
specifically, what your research question is, and what you hypothesize
will happen.

\begin{itemize}
\tightlist
\item
  The summary is the heart of the proposal, and requires some care. You
  will need to convince me that your topic is interesting, your data are
  relevant, and building a model and making predictions of a
  quantitative outcome using the predictors available to you will be
  worthwhile.
\item
  The summary ends with a statement of the research question or
  questions (you may have one, or possibly two.) An excellent question
  conveys the main objective of the study in terms that allow us to
  apply statistical models to describe an association between one or
  more predictors and a quantitative outcome.
\item
  It should be possible for me to explain your study accurately just by
  reading this summary. If it's not possible, it will come back to you
  for speedy rework.
\item
  This summary should be less than 300 words.
\item
  Use complete English sentences. Write in plain language. Use words we
  all know. Avoid jargon.
\end{itemize}

\hypertarget{the-data-description}{%
\subsection{The Data Description}\label{the-data-description}}

Your data description can be as long as it needs to be. It should
include:

\begin{enumerate}
\def\labelenumi{\arabic{enumi}.}
\item
  Your data source, which can be an online source (in which case include
  a working link), a published paper or journal article (in which case I
  need a link and a PDF copy of the paper), or unpublished data (in
  which case I need the details of how the data were gathered)
\item
  A thorough description of the data collection process, with complete
  details as to the nature of the variables, the setting for data
  collection, and complete details of any apparatus you used which may
  affect results.
\item
  Specification of the people and methods involved.

  \begin{itemize}
  \tightlist
  \item
    Who are the subjects under study?
  \item
    When were the data gathered? By whom?
  \item
    How many subjects are included?
  \item
    What caused subjects to be included or excluded from the study?
  \end{itemize}
\item
  Your planned quantitative outcome, which must relate directly to the
  research question you specified above. Provide a complete definition,
  including specifying the exact wording of the question or details of
  the measurement procedure used to obtain the outcome. If available,
  you can also include descriptios of secondary quantitative outcomes.
\item
  Your predictors of interest, which should also relate to the research
  question in an obvious way. Again, define the variables carefully, as
  you did with the outcome.
\item
  If you already have the data, tell me that. If you don't, specify any
  steps you must still take in order to get the data, and specify the
  date by which you will have your data (must be no later than
  2016-11-01.)
\end{enumerate}

\hypertarget{some-potentially-useful-data-sources}{%
\subsection{Some Potentially Useful Data
Sources}\label{some-potentially-useful-data-sources}}

The ideal choice of data source for this project is a public-use version
of a meaningful data set without access restrictions. With so many
students in the class, I cannot be responsible for supervising your work
with restricted data personally. Some appealing sources to explore
include:

\begin{itemize}
\tightlist
\item
  \url{https://www.data.gov/} The home of the U.S. Government's open
  data
\item
  \url{http://www.census.gov/data.html} The U.S. Census Bureau has many
  interesting data sets, including the
  \href{http://www.census.gov/programs-surveys/cps.html}{Current
  Population Survey}
\item
  \url{http://www.healthdata.gov/} 125 years of U.S. Health Care Data
\item
  \url{http://www.cdc.gov/nchs/nhanes/index.htm} National Health and
  Nutrition Examination Survey.

  \begin{itemize}
  \tightlist
  \item
    You may want to look at
    \href{https://cran.r-project.org/web/packages/nhanesA/vignettes/Introducing_nhanesA.html}{the
    nhanesA package in R}
  \end{itemize}
\item
  \url{http://dashboard.healthit.gov/datadashboard/data.php} Office of
  the National Coordinator for Health IT's dashboard
\item
  \url{http://www.icpsr.umich.edu/icpsrweb/} ICSPR (Inter-university
  Consortium for Political and Social Research) is a source for many
  public-use data sets

  \begin{itemize}
  \tightlist
  \item
    This includes the
    \href{http://www.icpsr.umich.edu/icpsrweb/HMCA/}{Health and Medical
    Care data archive of the Robert Wood Johnson Foundation}
  \end{itemize}
\item
  \url{http://gss.norc.org/} The General Social Survey
\item
  \url{http://www.bls.gov/data/} Bureau of Labor Statistics
\item
  \url{http://nces.ed.gov/surveys/} National Center for Education
  Statistics
\item
  \url{http://www.odh.ohio.gov/healthstats/dataandstats.aspx} Ohio
  Department of Health
\item
  \url{http://open.canada.ca/en} Canada Open Data
\item
  \url{http://digital.nhs.uk/home} Health data sets from the UK National
  Health Service.
\item
  \url{http://www.who.int/en/} World Health Organization
\item
  \url{http://www.unicef.org/statistics/} UNICEF has some available data
  on women and children
\item
  \url{http://www.pewinternet.org/datasets/} Pew Research Center's
  Internet Project
\item
  \url{http://portals.broadinstitute.org/cgi-bin/cancer/datasets.cgi}
  Broad Institute's Cancer Program
\item
  \url{http://www.kdnuggets.com/datasets/index.html} is a big index of
  lots of available data repositories
\item
  \url{https://www.kaggle.com/} Kaggle competition data sets are an
  interesting possibility
\end{itemize}

I cannot guarantee the quality of any of the data sets available at
these sites, but I've spent at least a little time at most of them in
recent months.

\hypertarget{some-restrictions}{%
\subsection{Some Restrictions}\label{some-restrictions}}

Note that it is especially appealing, in Study 2, to make use of data
that you are studying in your own field that fit the criteria I describe
and which can be made available to me. Ideally, therefore, you would be
working with data that are available to the public.

\begin{enumerate}
\def\labelenumi{\arabic{enumi}.}
\item
  You need to be able to share your data with a statistician (Dr.~Love)
  following \href{https://github.com/jtleek/datasharing}{Jeff Leek's
  guide to sharing data with a statistician}. This means you need to
  have access to the data in the raw, and it means that I have to be
  able to have access to it in the raw, as well. So, studies involving
  protected health information are \textbf{not} appropriate.
\item
  A full citation for any and all data elements, including a complete
  codebook, must be provided as part of your proposal.
\item
  There is more to a statistical application than the analysis of a
  canned data set, even a good canned data set. Googling ``data sets for
  regression projects'' or something similar is not a good strategy. I
  am not interested in you using pre-cleaned data from an educational
  repository, such as:

  \begin{itemize}
  \tightlist
  \item
    \href{http://www.lerner.ccf.org/qhs/datasets/}{this one at the
    Cleveland Clinic}, or
    \href{http://biostat.mc.vanderbilt.edu/wiki/Main/DataSets}{this one
    at Vanderbilt University}, or
    \href{http://www.stat.ucla.edu/projects/datasets/}{this one at
    UCLA}, or \href{http://www.stat.ufl.edu/~winner/datasets.html}{this
    one at the University of Florida}, or
    \href{http://people.sc.fsu.edu/~jburkardt/datasets/datasets.html}{this
    one at Florida State University}, or
  \item
    \href{http://lib.stat.cmu.edu/datasets/}{StatLib at Carnegie-Mellon
    University}, or
    \href{http://www.amstat.org/publications/jse/jse_data_archive.htm}{the
    Journal of Statistics Education Data Archive}, or
  \item
    \href{http://www.statsci.org/datasets.html}{StatSci.org's repository
    of textbook examples and ready for teaching data}, or
  \item
    any of the many textbook-linked repositories of data sets, like
    \href{http://www.lock5stat.com/datapage.html}{this one for
    Statistics: Unlocking the Power of Data}, or
  \item
    any similar repository Professor Love deems to be inappropriate
  \end{itemize}
\item
  While there are some great resources available to some people in this
  class by virtue of their affiliation with one of the health systems in
  town, I can do nothing to get you access to health system specific
  data as part of your project for this class or for 432, and in
  general, data from those sources are not especially appropriate
  because of issues with protected health information.
\end{enumerate}

\hypertarget{evaluating-task-a}{%
\section{Evaluating Task A}\label{evaluating-task-a}}

Dr.~Love will evaluate all proposals (Task A) personally, in the order
in which they are received. Proposals will receive one of two grades: OK
or REDO. REDO will be accompanied with specific requests that should be
accomplished within a short time window (approximately 24 hours). If you
materially deviate from these specifications, he will return your
proposal without comment other than to re-specify what needs to be fixed
before he responds.

\hypertarget{taskB}{%
\chapter{Task B (Presentation Sign-Up) Instructions}\label{taskB}}

\hypertarget{deadline-and-submission-information-1}{%
\section{Deadline and Submission
information}\label{deadline-and-submission-information-1}}

Task B is due at noon on 2018-10-12. Submit your Task B work by
completing the Google Form linked at \textbf{LINK GOES HERE}. Please
note that Task A is also due at the same time.

\hypertarget{taskC}{%
\chapter{Task C (Survey Editing) Instructions}\label{taskC}}

\hypertarget{deadline-and-submission-information-2}{%
\section{Deadline and Submission
information}\label{deadline-and-submission-information-2}}

Task C is due at noon on 2018-10-23. Submit your Task C work via
\href{https://canvas.case.edu/}{Canvas}. Please note that:

\begin{itemize}
\tightlist
\item
  Task D is also due at the same time.
\item
  We do not have class on 2018-10-23 because of CWRU's Fall Break.
\end{itemize}

You need to submit via Canvas, a single Word document (maximum one page,
12 point font, with your name and Project Task B on the top of the Word
document) containing these two things:

\begin{enumerate}
\def\labelenumi{\arabic{enumi}.}
\tightlist
\item
  Your list of typographical errors, clarifications or other edits to
  the 96 items currently included in the Survey Draft, available at
  \url{https://sites.google.com/a/case.edu/love-431/home/projects/class-survey}.
  If you found no errors or items in need of clarification, write a
  sentence saying that. If you did find an issue, please be sure to
  specify the item number (1-96) where you feel a revision is needed.

  \begin{itemize}
  \tightlist
  \item
    If you see any items in these 96 that you, personally, are not
    comfortable answering, \textbf{please indicate that to us} in this
    list, and we will consider revisions appropriately.
  \end{itemize}
\item
  Your list of 0-3 new
  items\footnote{We will not consider more than 3 new items from anyone, and are eager to hold the total set of new items to 25 or less, across all 66 students. I would argue that data related to each of the 66 accepted project proposals may be found in the existing set of 96 items.}
  that you would like to add to the survey.

  \begin{itemize}
  \tightlist
  \item
    Note that your new items \emph{can} be but do not \emph{need} to be
    anything you've previously suggested.
  \item
    Please begin with the following sentence:
    \texttt{I\ would\ like\ to\ submit\ \#\ new\ items\ for\ consideration.}
  \item
    If your number of new items to suggest is zero, then you need not
    write anything else here.
  \item
    Should you wish to have us include 1-3 additional items, please
    remember that nothing about sex, drugs, or performance in 431 can be
    asked, and:

    \begin{enumerate}
    \def\labelenumii{\alph{enumii}.}
    \tightlist
    \item
      list each new item, being sure to specify the type (for instance,
      short answer, multiple choice, or checkbox) and the set of
      possible responses, as you did in the proposal.
    \item
      describe (in 2-3 complete sentences per new item) your reasons to
      include the item.

      \begin{itemize}
      \tightlist
      \item
        Good reasons would begin with a statement of what you intend to
        do. As an example of such a statement, consider
        \texttt{I\ wish\ to\ study\ the\ result\ of\ this\ new\ item\ as\ a\ quantitative\ outcome\ across\ groups\ established\ by\ current\ item\ \#***\ from\ the\ survey.}
        Or, perhaps, something like:
        \texttt{I\ wish\ to\ use\ this\ new\ item\ as\ a\ grouping\ variable\ to\ study\ current\ item\ \#***.}
      \item
        In either case, follow your statement with a short explanation
        as to why your new item's result is of interest, and is not
        already captured by the existing survey.
      \end{itemize}
    \end{enumerate}
  \end{itemize}
\end{enumerate}

\hypertarget{taskD}{%
\chapter{Task D (Survey Comparison Plan) Instructions}\label{taskD}}

\hypertarget{deadline-and-submission-information-3}{%
\section{Deadline and Submission
information}\label{deadline-and-submission-information-3}}

Task D is due at noon on 2018-10-23. Submit your Task D work via
\href{https://canvas.case.edu/}{Canvas}. Please note that:

\begin{itemize}
\tightlist
\item
  Task C is also due at the same time.
\item
  We do not have class on 2018-10-23 because of CWRU's Fall Break.
\end{itemize}

\begin{enumerate}
\def\labelenumi{\arabic{enumi}.}
\tightlist
\item
  \textbf{A Word document submitted via Blackboard}, specifying the list
  of items from the survey that you want to be able to use in your
  analysis, using the \textbf{Template for Task C} available at
  \url{https://sites.google.com/a/case.edu/love-431/home/projects/class-survey}.

  \begin{itemize}
  \tightlist
  \item
    You need not do any analyses connected to the items you originally
    suggested, nor do you need to do analyses that mirror your original
    research questions.
  \item
    The Template asks you to specify (by item number and name) the items
    you wish to you use in your analyses, for each of the six analyses
    you will complete for Study 1.
  \item
    Task E has details. You need to complete either Analysis 1a or 1b,
    and then Analyses 2-6.
  \item
    In addition to the items you select related to each Analysis, you
    will also select two backup quantitative variables, and two backup
    factors, as described in the Template.
  \item
    Items with at least 10 possible responses will be treated as
    quantitative. Other items will be treated as categorical (factors.)
    For ordered categories, you can consider assigning a score to each
    response, then treating that score as quantitative.

    \begin{itemize}
    \tightlist
    \item
      You are permitted to categorize any quantitative item you choose.
    \item
      You are permitted to collapse any categories in an item with more
      than 2 categories, as you choose.
    \item
      Some items are part of multiple-item scales. If you want to use a
      scale, specify each item that would go into that scale in Task C.
    \end{itemize}
  \end{itemize}
\end{enumerate}

\begin{longtable}[]{@{}rl@{}}
\toprule
\begin{minipage}[b]{0.31\columnwidth}\raggedleft
Analysis\strut
\end{minipage} & \begin{minipage}[b]{0.63\columnwidth}\raggedright
Variables needed\strut
\end{minipage}\tabularnewline
\midrule
\endhead
\begin{minipage}[t]{0.31\columnwidth}\raggedleft
{[}1a{]} 2 means via paired samples\strut
\end{minipage} & \begin{minipage}[t]{0.63\columnwidth}\raggedright
Two quantitative (outcomes)\strut
\end{minipage}\tabularnewline
\begin{minipage}[t]{0.31\columnwidth}\raggedleft
{[}1b{]} 2 means via indep. samples\strut
\end{minipage} & \begin{minipage}[t]{0.63\columnwidth}\raggedright
One quantitative (outcome) and one categorical (2 levels)\strut
\end{minipage}\tabularnewline
\begin{minipage}[t]{0.31\columnwidth}\raggedleft
{[}2{]} ANOVA with Tukey\strut
\end{minipage} & \begin{minipage}[t]{0.63\columnwidth}\raggedright
One quantitative (outcome) and one categorical (3-6 levels)\strut
\end{minipage}\tabularnewline
\begin{minipage}[t]{0.31\columnwidth}\raggedleft
{[}3{]} Regression Model\strut
\end{minipage} & \begin{minipage}[t]{0.63\columnwidth}\raggedright
Same as either {[}1b{]} or {[}2{]}, plus one quantitative
(covariate)\strut
\end{minipage}\tabularnewline
\begin{minipage}[t]{0.31\columnwidth}\raggedleft
{[}4{]} 2x2 Table\strut
\end{minipage} & \begin{minipage}[t]{0.63\columnwidth}\raggedright
Two categorical (2 level) variables\strut
\end{minipage}\tabularnewline
\begin{minipage}[t]{0.31\columnwidth}\raggedleft
{[}5{]} JxK Table\strut
\end{minipage} & \begin{minipage}[t]{0.63\columnwidth}\raggedright
Two categorical variables, one with 2-6, other with 3-6 levels\strut
\end{minipage}\tabularnewline
\begin{minipage}[t]{0.31\columnwidth}\raggedleft
{[}6{]} 2x2xJ Table\strut
\end{minipage} & \begin{minipage}[t]{0.63\columnwidth}\raggedright
Same as {[}4{]}, plus one categorical with 3-6 levels\strut
\end{minipage}\tabularnewline
\bottomrule
\end{longtable}

\hypertarget{taskE}{%
\chapter{Task E (Taking the Survey) Instructions}\label{taskE}}

\hypertarget{deadline-and-submission-information-4}{%
\section{Deadline and Submission
information}\label{deadline-and-submission-information-4}}

Task E is due at noon on 2018-10-31. Submit your answers to the course
survey via the Google Form linked at \textbf{LINK GOES HERE}. That link
will go live after class on 2018-10-25.

\begin{enumerate}
\def\labelenumi{\arabic{enumi}.}
\setcounter{enumi}{1}
\tightlist
\item
  \textbf{Completion (Google form) of the final course survey},
  available on November 7 by 5 PM.

  \begin{itemize}
  \tightlist
  \item
    The final item asks for your name, and the system is collecting your
    email address (you must be logged into Google via CWRU). These will
    be pruned from the survey before data sets are created.
  \item
    You should answer all of the items. Please don't skip any items you
    can answer. Your colleagues need data.
  \item
    If you want to save your work and return later, note that only the
    \emph{first} item in each section of the survey must be completed
    for Google to let you submit your work. Once you've submitted a
    partially completed survey, you can return as often as you like
    before the deadline to finish up.
  \end{itemize}
\end{enumerate}

\hypertarget{receiving-your-study-1-data-november-18}{%
\section{Receiving Your Study 1 Data (November
18)}\label{receiving-your-study-1-data-november-18}}

\begin{itemize}
\tightlist
\item
  We will post \textbf{two} data files for you, each containing some of
  the variables you need.
\item
  You will need to download both files, and then \emph{combine} and tidy
  to suit your needs.
\item
  The two files will be linked by the subject \texttt{id} number.
\item
  We discuss combining two data sets, using \texttt{dplyr}, as part of
  the Data Management Tips section.
\end{itemize}

\hypertarget{taskF}{%
\chapter{Task F (Sharing Study 2 Data) Instructions}\label{taskF}}

\hypertarget{deadline-and-submission-information-5}{%
\section{Deadline and Submission
information}\label{deadline-and-submission-information-5}}

Task F is due at noon on 2018-11-14. Submit your Task F work through
\href{https://canvas.case.edu/}{Canvas}.

Task F requires you to share your data for Study 2. The model for this
Task is Jeff Leek's \href{https://github.com/jtleek/datasharing}{Guide
to Data Sharing}. Specifically, Task D requires that you submit the
following to Dr.~Love by the deadline.

\begin{enumerate}
\def\labelenumi{\arabic{enumi}.}
\tightlist
\item
  a direct link to the raw data set (without any need for me to sign up
  for anything) or a .csv copy of the raw data set called
  \texttt{yourname-raw.csv}
\item
  a single .csv file with a name of your choice containing a clean, tidy
  data set for Study 2, along with
\item
  a Word or PDF file containing both

  \begin{enumerate}
  \def\labelenumii{\alph{enumii}.}
  \tightlist
  \item
    a \textbf{codebook} section which describes every variable (column)
    and its values in your .csv file,
  \item
    a \textbf{study design} section which reminds (and updates) us about
    the source of the data and your research question.
  \end{enumerate}
\end{enumerate}

\hypertarget{the-raw-data-set}{%
\section{The Raw Data Set}\label{the-raw-data-set}}

You need to show me the raw, de-identified data. The data are raw if
you:

\begin{itemize}
\tightlist
\item
  Ran no software on the data and Did not manipulate any of the numbers
  in the data
\item
  You did not remove any data from the data set other than to
  de-identify it and eliminate protected information and anything else
  that you cannot share
\item
  You did not summarize the data in any way
\end{itemize}

A direct link (without me having to sign up for anything) is preferred.
If this is not possible, send a .csv file of the raw data set, called
\texttt{yourname-raw.csv}. Note that you should not send me any
variables you have no chance of using in your analyses, but may include
some variables you haven't made a final decision on.

\hypertarget{the-tidy-data-set}{%
\section{The Tidy Data Set}\label{the-tidy-data-set}}

Your .csv file should include only those variables you will actually use
in your analysis of Study 2. Your .csv file should include one row per
subject in your data, and one column for each variable you will use.
Your data are tidy if each variable you measure is in its own column,
and each different observation of that variable is in its own row,
identifed by the subject \texttt{id}.

You need to provide:

\begin{enumerate}
\def\labelenumi{\arabic{enumi}.}
\tightlist
\item
  a header row (row 1 in the spreadsheet) that contains full row names.
  So if you measured age at diagnosis for patients, you would head that
  column with the name \texttt{AgeAtDiagnosis} or
  \texttt{Age.at.Diagnosis} instead of something like \texttt{ADx} or
  another abbreviation that may be hard for another person (or you, two
  years from now) to understand.
\item
  a study identification number (I would call this variable \texttt{id}
  and use consecutive integers to represent the rows in your data set)
  which should be the left-most variable in your tidy data.
\item
  a quantitative outcome with a meaningful name using no special
  characters other than a period (\texttt{.}), hyphen(\texttt{-}) or
  underscore (\texttt{\_}) used to separate words, which should be the
  second variable in your data.

  \begin{itemize}
  \tightlist
  \item
    If you have any missing \textbf{outcome} values, \textbf{delete
    those rows} entirely from your tidy data set.
  \end{itemize}
\item
  at least four predictor variables, each with a meaningful name using
  no special characters other than \texttt{.} or \texttt{\_} to separate
  words, and the predictors should be shown in columns to the right of
  the outcome.

  \begin{itemize}
  \tightlist
  \item
    \emph{Continuous} variables are anything measured on a quantitative
    scale that could be any fractional number.
  \item
    \emph{Ordinal categorical} data are data that have a fixed, small
    (\textless{} 100) number of levels but are ordered.
  \item
    \emph{Nominal categorical} data are data where there are multiple
    categories, but they aren't ordered.
  \item
    Categorical predictors should read into R as factors, so your
    categories should include letters, and not just numbers. In general,
    try to avoid coding nominal or ordinal categorical variables as
    numbers.
  \item
    Label your categorical predictors in the way you plan to use them in
    your analyses
  \item
    \emph{Missing data} are data that are missing and you don't know the
    mechanism. Missing data in the predictor variables are allowed, and
    you should code missing values in your tidy data set as \texttt{NA}.
    It is critical to report if there is a reason you know about that
    some of the data are missing. You should also not impute/make
    up/throw away missing observations on the predictor values in your
    tidy data set.
  \end{itemize}
\item
  any other variables you need to share with me (typically this would
  only include things you had to use in order to get to your final
  choice of outcome and predictors.) Most people will not need to share
  any additional variables.
\end{enumerate}

I will need to be able to take your submitted \texttt{.csv} file and run
your eventual Markdown file (Task E) against it and obtain your results,
so it must be completely clean. Because it is a \texttt{.csv} file,
you'll have no highlighting or bolding or any other special formatting.
If you have missing values, they should be indicated as NA in the file.
If you obtain the file in R, and then write it to a .csv file, you
should write the file without row numbers if you already have an
identification variable. To do so, you should be able to use
\texttt{write.csv(dataframeinR,\ "newfilename.csv",\ row.names\ =\ FALSE)}
where you will substitute in the name of your data frame in R, and new
(.csv) file name. Don't use the same name for your original data set and
your tidy one.

\hypertarget{the-codebook}{%
\section{The Codebook}\label{the-codebook}}

For almost any data set, the measurements you calculate will need to be
described in more detail than you will sneak into the spreadsheet. The
code book contains this information. At minimum it should contain:

\begin{enumerate}
\def\labelenumi{\arabic{enumi}.}
\tightlist
\item
  Information about the variables (including units! and codes for any
  categorical variables) in your tidy data set
\item
  Information about the summary choices or transformations you made or
  the development of any scales from raw data
\end{enumerate}

By reading the codebook, I should understand what you did to get from
the raw data to your tidy data, so add any additional information you
need to provide to make that clear.

\hypertarget{the-study-design}{%
\section{The Study Design}\label{the-study-design}}

Here is where I want you to put the information about the experimental
study design you used. You can and should reuse (and edit) the
information you provided as part of the Proposal in this Codebook. The
material you need here consists of three parts from the proposal,
updated to mirror your current plan. Specifically, you should provide:

\begin{enumerate}
\def\labelenumi{\arabic{enumi}.}
\tightlist
\item
  Your research question describes your outcome, your key predictor and
  other predictors, and the population of interest. It is probably
  easiest to follow one of these
  formats\footnote{You are welcome to move the clauses around to make for a clearer question.}.
\end{enumerate}

\begin{itemize}
\tightlist
\item
  What is the effect of \texttt{*your\ key\ predictor*} on
  \texttt{*your\ outcome*} adjusting for
  \texttt{*your\ list\ of\ other\ predictors*} in
  \texttt{*your\ population\ of\ subjects*}?
\item
  How effectively can \texttt{*specify\ your\ predictors*} predict
  \texttt{*your\ outcome*} in \texttt{*your\ population\ of\ subjects*}?
  or
\end{itemize}

\begin{enumerate}
\def\labelenumi{\arabic{enumi}.}
\setcounter{enumi}{1}
\item
  A thorough description of the data collection process, with complete
  details as to the nature of the variables, the setting for data
  collection, and complete details of any apparatus you used which may
  affect results that \textbf{has not already been covered} in the
  codebook materials.
\item
  Specification of the people and methods involved.

  \begin{enumerate}
  \def\labelenumii{\alph{enumii}.}
  \tightlist
  \item
    Who are the subjects under study? How many are included in your
    final tidy data set?
  \item
    When were the data gathered? By whom?
  \item
    What caused subjects to be included or excluded from the study?
  \end{enumerate}
\end{enumerate}

\hypertarget{taskG}{%
\chapter{Task G (The Project Update) Instructions}\label{taskG}}

\hypertarget{deadline-and-submission-information-6}{%
\section{Deadline and Submission
information}\label{deadline-and-submission-information-6}}

Task G is due at noon on 2018-11-28. Submit your Task G work through
\href{https://canvas.case.edu/}{Canvas}.

\hypertarget{taskH}{%
\chapter{Task H (The Portfolio) Instructions}\label{taskH}}

Task H requires you to provide a written portfolio of materials, which
you will also make use of in your final presentation.

\hypertarget{logistics}{%
\section{Logistics}\label{logistics}}

\begin{itemize}
\tightlist
\item
  Submit your portfolio via email to Dr.~Love. Make your email's
  subject: \texttt{431\ TASK\ E\ FOR\ YOUR\ NAME}.
\item
  The portfolio should be contained in a single .zip file that contains
  each of the elements below.
\item
  Name your .zip file YOURNAME-TASK E.zip
\item
  The .zip file should contain

  \begin{itemize}
  \tightlist
  \item
    (for Study 1) a .csv, a .Rmd and a .doc/.docx/.pdf file generated
    from that .Rmd file, and
  \item
    (for Study 2) a .Rmd and a .doc/.docx/.pdf file generated from that
    .Rmd file.
  \end{itemize}
\end{itemize}

\hypertarget{materials-for-task-e}{%
\section{Materials for Task E}\label{materials-for-task-e}}

\begin{itemize}
\tightlist
\item
  {[}\texttt{.csv} file{]} A clean, tidy data set for Study 1, which
  will require combining the two data sets you are provided, dealing
  with any missing data and any necessary combination into scales on the
  variables in which you are interested.
\item
  {[}\texttt{.Rmd} and \texttt{.doc/.docx/.pdf} files{]} The six
  required analyses for Study 1, as both a Markdown file and Word/PDF
  that work with the clean and tidy data set for Study 1.
\item
  {[}\texttt{.Rmd} and \texttt{.doc/.docx/.pdf} files{]} The eight
  required analyses for Study 2, as both a Markdown file and Word/PDF
  that work with the clean and tidy data set you submitted in Task D.
\end{itemize}

\hypertarget{the-six-required-analyses-for-study-1-the-survey}{%
\section{The Six Required Analyses for Study 1 (The
Survey)}\label{the-six-required-analyses-for-study-1-the-survey}}

The required analyses for the Project Survey that need to be in your
Portfolio are:

\begin{enumerate}
\def\labelenumi{\arabic{enumi}.}
\tightlist
\item
  A two-group comparison of population means (could use paired or
  independent samples)
\item
  An analysis of variance with Tukey HSD pairwise comparisons of
  population means across K subgroups, where 3 \(\leq\) K \textless{} 7
\item
  A regression model to amplify the indepedent samples comparison in a
  or b by incorporating a quantitative covariate.
\item
  A 2x2 Table and resulting analyses for comparison of two population
  proportions in terms of relative risk, odds ratio and probability
  difference
\item
  A two-way JxK contingency table where 2 \(\leq\) J \textless{} 7 and 3
  \(\leq\) K \textless{} 7 with an appropriate chi-square test
\item
  A three way 2 x 2 x J contingency table analysis whch will expand your
  2x2 table from \#4 and where 3 \(\leq\) J \textless{} 7
\end{enumerate}

A demonstration of an appropriate analysis for each of these pieces will
be provided at
\url{https://sites.google.com/a/case.edu/love-431/home/projects/class-survey}.

\newpage

\hypertarget{the-eight-required-analyses-for-study-2-your-data}{%
\section{The Eight Required Analyses for Study 2 (Your
Data)}\label{the-eight-required-analyses-for-study-2-your-data}}

For your portfolio presentation in Study 2 (Your Data) complete these
steps:

\begin{enumerate}
\def\labelenumi{\arabic{enumi}.}
\setcounter{enumi}{-1}
\tightlist
\item
  Identify all the variables in your tidy data set that have missing
  (NA) values. Delete all observations with missing outcomes, and use
  simple imputation to impute values for the candidate predictors with
  NAs. Use the resulting imputed data set in all subsequent work.
\item
  Obtain a training sample with a randomly selected 80\% of your data,
  and have the remaining 20\% in a test sample, properly labeled, and
  using \texttt{set.seed} so that the results can be replicated later.
  Use this training sample for Steps 2-6 below.
\item
  Using the training sample, provide numerical summaries of each
  predictor variable and the outcome (with \texttt{Hmisc::describe}), as
  well as graphical summaries of the outcome variable. Your results
  should now show no missing values in any variable. Are there any
  evident problems, such as substantial skew in the outcome variable?
\item
  Build and interpret a scatterplot matrix to describe the associations
  (both numerically and graphically) between the outcome and all
  predictors. Use a Box-Cox plot to investigate whether a transformation
  of your outcome is suggested. Describe what a correlation matrix
  suggests about collinearity between candidate predictors.
\item
  Specify a ``kitchen sink'' linear regression model to describe the
  relationship between your outcome (potentially after transformation)
  and the main effects of each of your predictors. Assess the overall
  effectiveness, within your training sample, of your model, by
  specifying and interpreting the R\textsuperscript{2}, adjusted
  R\textsuperscript{2} (especially in light of your collinearity
  conclusions below), the residual standard error, and the ANOVA F test.
  Does collinearity in the kitchen sink model have a meaningful impact?
  How can you tell? Specify the size, magnitude and meaning of all
  coefficients, and identify appropriate conclusions regarding effect
  sizes with 90\% confidence intervals.
\item
  Build a second linear regression model using a subset of your four
  predictors, chosen by you to maximize predictive value within your
  training sample. Specify the method you used to obtain this new model.
  (Backwards stepwise elimination is a likely approach in many cases,
  but if that doesn't produce a new model, feel free to select two of
  your more interesting predictors from the kitchen sink model and run
  that as a new model.)
\item
  Compare this new (second) model to your ``kitchen sink'' model within
  your training sample using adjusted R\textsuperscript{2}, the residual
  standard error, AIC and BIC. Specify the complete regression equation
  in both models, based on the training sample. Which model appears
  better in these comparisons of the four summaries listed above?
  Produce a table to summarize your results. Does one model ``win'' each
  competition in the training sample?
\item
  Now, use your two regression models to predict the value of your
  outcome using the predictor values you observe in the test sample. Be
  sure to back-transform the predictions to the original units if you
  wound up fitting a model to a transformed outcome. Compare the two
  models in terms of mean squared prediction error and mean absolute
  prediction error in a Table, which Dr.~Love will \textbf{definitely
  want to see} in your portfolio. Which model appears better at
  out-of-sample prediction according to these comparisons, and how do
  you know?
\item
  Select the better of your two models (based on the results you obtain
  in Questions 6 and 7) and apply it to the entire data set. Do the
  coefficients or summaries the model show any important changes when
  applied to the entire data set, and not just the training set? Plot
  residuals against fitted values, and also a Normal probability plot of
  the residuals, each of which Dr.~Love \textbf{will be looking for} in
  your portfolio. What do you conclude about the validity of standard
  regression assumptions for your final model based on these two plots?
\end{enumerate}

A demonstration of an appropriate analysis for each of these pieces is
available at
\url{https://sites.google.com/a/case.edu/love-431/home/projects/your-data}

\hypertarget{taskI}{%
\chapter{Task I (Your Presentation) Instructions}\label{taskI}}

The presentation schedule is found at \url{https://goo.gl/PivgQx}.
Arrive at Dr.~Love's office (Wood WG-82L) at least 5 minutes early. If
the door is open, please be sure that Dr.~Love knows you are there.

You will give your final presentation in a 15-minute meeting with
Dr.~Love. This will involve materials from both of your studies, in a
fairly regimented way, described below.

\begin{itemize}
\tightlist
\item
  You are welcome to bring either a printed presentation or (better) a
  functioning laptop which you can use to show me the key results as you
  describe them for each of the analyses in Study 1 and in Study 2 that
  you wind up discussing.
\item
  You are welcome to show me results in the context of a
  Powerpoint-style presentation, if you prefer to develop one, or to
  show me results straight from your Markdown-created Word or PDF files
  in your portfolio. Whatever works for you - so long as I can see what
  you are talking about as you are talking, we'll be fine.
\item
  The computers in my office will be busy while we are meeting, so I
  will NOT be able to pull up your portfolio or data while we are
  talking. You will have to be able to do that.
\item
  It is 100\% appropriate for you to ask questions before the
  presentation, of Dr.~Love or the TAs. Please do. At the presentation,
  there will be a little time for Dr.~Love to address any lingering
  questions after the main presentation, and he's eager to hear your
  questions at that time, too.
\end{itemize}

\hypertarget{study-1-presentation-about-5-minutes-total}{%
\section{Study 1 Presentation (about 5 minutes,
total)}\label{study-1-presentation-about-5-minutes-total}}

In Study 1, you will first select your most interesting / intriguing
result out of your six main analyses and present that, in about 90
seconds. In those 90 seconds, you should be showing me the highlights,
specifically:

\begin{enumerate}
\def\labelenumi{\alph{enumi}.}
\tightlist
\item
  What question were you investigating?
\item
  What conclusion did you draw about that question?
\item
  What statistical method led you to that conclusion?
\end{enumerate}

I will then ask you to present the results of one of the other five main
analyses, in a similar way. You will need to come prepared to present
this information for any of your six Study 1 analyses at a moment's
notice, as you will not know in advance which of your other five main
analyses I will ask for.

\hypertarget{study-2-presentation-about-10-minutes-total}{%
\section{Study 2 Presentation (about 10 minutes,
total)}\label{study-2-presentation-about-10-minutes-total}}

In Study 2, you will start with telling me about the most important
finding of your little study in four minutes. In these 4 minutes, you
will tell me:

\begin{enumerate}
\def\labelenumi{\alph{enumi}.}
\tightlist
\item
  What your research question was
\item
  Why it was interesting to you (parts 1 and 2 combined should take no
  more than 30 seconds)
\item
  What your better model has to say about the answer to your research
  question

  \begin{itemize}
  \tightlist
  \item
    This should include a description of the predictors that wound up in
    your (final) model and the direction of each of their effects on
    your outcome. Show me the model as you're telling me about this.
  \item
    This should also include a sense of how well the model predicted
    overall (R\textsuperscript{2} is one good choice)
  \item
    This should also include how well the residual plots for your final
    model fit regression assumptions. Show me the plots as you're
    telling me about this.
  \end{itemize}
\item
  Your conclusions about rational next steps to learn more from these
  data, or what specific new data you now wish you'd had when you
  started the study.
\end{enumerate}

For most of the remaining time, I will ask you about your study, and try
to help you think through any problems you had in obtaining or
interpreting analyses. You should come prepared to share any of the 8
steps of your analysis at a moment's notice, as we may want to look at
any part of your work.

\bibliography{book.bib,packages.bib}


\end{document}
