\documentclass[]{book}
\usepackage{lmodern}
\usepackage{amssymb,amsmath}
\usepackage{ifxetex,ifluatex}
\usepackage{fixltx2e} % provides \textsubscript
\ifnum 0\ifxetex 1\fi\ifluatex 1\fi=0 % if pdftex
  \usepackage[T1]{fontenc}
  \usepackage[utf8]{inputenc}
\else % if luatex or xelatex
  \ifxetex
    \usepackage{mathspec}
  \else
    \usepackage{fontspec}
  \fi
  \defaultfontfeatures{Ligatures=TeX,Scale=MatchLowercase}
\fi
% use upquote if available, for straight quotes in verbatim environments
\IfFileExists{upquote.sty}{\usepackage{upquote}}{}
% use microtype if available
\IfFileExists{microtype.sty}{%
\usepackage{microtype}
\UseMicrotypeSet[protrusion]{basicmath} % disable protrusion for tt fonts
}{}
\usepackage[margin=1in]{geometry}
\usepackage{hyperref}
\hypersetup{unicode=true,
            pdftitle={431 Project Instructions},
            pdfauthor={Thomas E. Love},
            pdfborder={0 0 0},
            breaklinks=true}
\urlstyle{same}  % don't use monospace font for urls
\usepackage{natbib}
\bibliographystyle{apalike}
\usepackage{longtable,booktabs}
\usepackage{graphicx,grffile}
\makeatletter
\def\maxwidth{\ifdim\Gin@nat@width>\linewidth\linewidth\else\Gin@nat@width\fi}
\def\maxheight{\ifdim\Gin@nat@height>\textheight\textheight\else\Gin@nat@height\fi}
\makeatother
% Scale images if necessary, so that they will not overflow the page
% margins by default, and it is still possible to overwrite the defaults
% using explicit options in \includegraphics[width, height, ...]{}
\setkeys{Gin}{width=\maxwidth,height=\maxheight,keepaspectratio}
\IfFileExists{parskip.sty}{%
\usepackage{parskip}
}{% else
\setlength{\parindent}{0pt}
\setlength{\parskip}{6pt plus 2pt minus 1pt}
}
\setlength{\emergencystretch}{3em}  % prevent overfull lines
\providecommand{\tightlist}{%
  \setlength{\itemsep}{0pt}\setlength{\parskip}{0pt}}
\setcounter{secnumdepth}{5}
% Redefines (sub)paragraphs to behave more like sections
\ifx\paragraph\undefined\else
\let\oldparagraph\paragraph
\renewcommand{\paragraph}[1]{\oldparagraph{#1}\mbox{}}
\fi
\ifx\subparagraph\undefined\else
\let\oldsubparagraph\subparagraph
\renewcommand{\subparagraph}[1]{\oldsubparagraph{#1}\mbox{}}
\fi

%%% Use protect on footnotes to avoid problems with footnotes in titles
\let\rmarkdownfootnote\footnote%
\def\footnote{\protect\rmarkdownfootnote}

%%% Change title format to be more compact
\usepackage{titling}

% Create subtitle command for use in maketitle
\newcommand{\subtitle}[1]{
  \posttitle{
    \begin{center}\large#1\end{center}
    }
}

\setlength{\droptitle}{-2em}

  \title{431 Project Instructions}
    \pretitle{\vspace{\droptitle}\centering\huge}
  \posttitle{\par}
    \author{Thomas E. Love}
    \preauthor{\centering\large\emph}
  \postauthor{\par}
      \predate{\centering\large\emph}
  \postdate{\par}
    \date{Version: 2018-09-18 07:50:50}

\usepackage{booktabs}
\usepackage{amsthm}
\makeatletter
\def\thm@space@setup{%
  \thm@preskip=8pt plus 2pt minus 4pt
  \thm@postskip=\thm@preskip
}
\makeatother

\usepackage{amsthm}
\newtheorem{theorem}{Theorem}[chapter]
\newtheorem{lemma}{Lemma}[chapter]
\theoremstyle{definition}
\newtheorem{definition}{Definition}[chapter]
\newtheorem{corollary}{Corollary}[chapter]
\newtheorem{proposition}{Proposition}[chapter]
\theoremstyle{definition}
\newtheorem{example}{Example}[chapter]
\theoremstyle{definition}
\newtheorem{exercise}{Exercise}[chapter]
\theoremstyle{remark}
\newtheorem*{remark}{Remark}
\newtheorem*{solution}{Solution}
\begin{document}
\maketitle

{
\setcounter{tocdepth}{1}
\tableofcontents
}
\hypertarget{overview}{%
\chapter*{Overview}\label{overview}}
\addcontentsline{toc}{chapter}{Overview}

This website contains the Fall 2018 project information for PQHS / CRSP
/ MPHP 431: Statistical Methods in Biological \& Medical Sciences,
Section 1.

\begin{itemize}
\tightlist
\item
  All materials related to the project (including these instructions)
  are maintained and linked at
  \url{https://github.com/THOMASELOVE/431-2018-project}.
\item
  The direct link to this document is
  \url{https://thomaselove.github.io/431-2018-project}.
\end{itemize}

\hypertarget{your-project-includes-two-studies}{%
\section*{Your Project includes Two
Studies}\label{your-project-includes-two-studies}}
\addcontentsline{toc}{section}{Your Project includes Two Studies}

Your final project for this course will result in a portfolio of work
related to two studies.

\textbf{Study 1 - Class Survey}. In the first study, you (sometimes
working individually, sometimes in a group) will design, administer,
analyze and present the results of a survey designed to compare two or
three groups of subjects on some \emph{categorical} and
\emph{quantitative} outcomes we will develop from your initial ideas.

\textbf{Study 2 - Your Data}. In the second study, you (working
individually) will propose a research question and relevant data of
interest to you, and then complete all elements of a data science
project designed to create a statistical model for a \emph{quantitative}
outcome, then use it for prediction and assess the quality of those
predictions.

\hypertarget{you-have-nine-tasks-to-complete-this-semester}{%
\section*{You have Nine Tasks to Complete this
Semester}\label{you-have-nine-tasks-to-complete-this-semester}}
\addcontentsline{toc}{section}{You have Nine Tasks to Complete this
Semester}

The project involves two analyses (one for the class survey and one for
your personal study), and a total of 9 tasks (deliverables.) Each task
is to be completed by \textbf{12 NOON} on the specified date.

\begin{itemize}
\tightlist
\item
  \protect\hyperlink{taskA}{Task A (The Proposal)} is due at noon on
  2018-10-12, via \href{https://canvas.case.edu/}{Canvas}.
\item
  \protect\hyperlink{taskB}{Task B (Presentation Sign-Up)} is
  \textbf{also} due at noon on 2018-10-12, at
  \url{http://bit.ly/431-2018-project-signup-taskB}
\item
  \protect\hyperlink{taskC}{Task C (Survey Editing)} involves group work
  and is due at noon on 2018-10-23, via
  \href{https://canvas.case.edu/}{Canvas}.
\item
  \protect\hyperlink{taskD}{Task D (Survey Comparison Plan)} is
  \textbf{also} due at noon on 2018-10-23, at
  \url{http://bit.ly/431-2018-survey-comparison-plan-taskD}.
\item
  \protect\hyperlink{taskE}{Task E (Taking the Survey)} is due at noon
  on 2018-10-31
\item
  \protect\hyperlink{taskF}{Task F (Sharing Study 2 Data)} is due at
  noon on 2018-11-14, via \href{https://canvas.case.edu/}{Canvas}.
\item
  \protect\hyperlink{taskG}{Task G (The Update)} is due at noon on
  2018-11-28.
\item
  \protect\hyperlink{taskH}{Task H (The Portfolio)} is due at noon on
  2018-12-13, via \href{https://canvas.case.edu/}{Canvas}.
\item
  \protect\hyperlink{taskI}{Task I (Your Presentation)} will be held on
  2018-12-10, 2018-12-11 or 2018-12-13 in Dr.~Love's office (Wood WG
  82-L).
\end{itemize}

The bulk of this document contains specific instructions for each of
these tasks.

\hypertarget{working-with-this-document}{%
\section*{Working with This Document}\label{working-with-this-document}}
\addcontentsline{toc}{section}{Working with This Document}

\begin{enumerate}
\def\labelenumi{\arabic{enumi}.}
\tightlist
\item
  This document is broken down into multiple sections. Use the table of
  contents at left to navigate.
\item
  At the top of the document, you'll see icons which you can click to

  \begin{itemize}
  \tightlist
  \item
    search the document,
  \item
    change the size, font or color scheme of the page, and
  \item
    download a PDF or EPUB (Kindle-readable) version of the entire
    document.
  \end{itemize}
\item
  The document is a work in progress, and will be updated occasionally
  through the semester. Check the Version information above to verify
  the last update time\footnote{Note that the ePub and PDF versions will
    show slightly different times (but on the same day) as the HTML
    version.}.
\end{enumerate}

\hypertarget{need-help}{%
\section*{Need Help?}\label{need-help}}
\addcontentsline{toc}{section}{Need Help?}

Questions about the project or the course can be directed to
\textbf{431-help at case dot edu} or to Dr.~Love directly at
\texttt{thomas\ dot\ love\ at\ case\ dot\ edu}.

\begin{itemize}
\tightlist
\item
  The course home page is at
  \url{https://github.com/THOMASELOVE/431-2018}
\end{itemize}

\hypertarget{project-objectives}{%
\chapter{Project Objectives}\label{project-objectives}}

It is hard to learn statistics (or anything else) passively; concurrent
theory and application are essential\footnote{Though by no means an
  original idea, this particular phrasing is stolen from Harry Roberts.}.

\hypertarget{study-1-is-about-making-comparisons-and-visualizing-groups-of-data.}{%
\section{Study 1 is about making comparisons and visualizing groups of
data.}\label{study-1-is-about-making-comparisons-and-visualizing-groups-of-data.}}

\textbf{Study 1} involves data from a \textbf{class survey}, to be
conducted in October. We will design, administer, analyze and present
survey results designed to compare two or three groups of subjects from
the class on some \emph{categorical} and \emph{quantitative} outcomes.
In the analysis stage, everyone will be working with different parts of
the same data set.

\begin{quote}
Think of a graph as a comparison. All graphs are comparisons (indeed,
all statistical analyses are comparisons). If you already have the graph
in mind, think of what comparisons it's enabling. Or if you haven't
settled on the graph yet, think of what comparisons you'd like to make.
\href{http://andrewgelman.com/2014/03/25/statistical-graphics-course-statistical-graphics-advice/}{Andrew
Gelman}
\end{quote}

In your eventual analysis of Study 1, you will be comparing both
quantitative and categorical outcomes across 2-3 groups. All tools
necessary for Study 1 are in Parts A and B of the course, and include
the following\ldots{}

\begin{itemize}
\tightlist
\item
  Descriptive and exploratory summaries of the data across the groups
  for each of your chosen outcomes, including, of course, attractive and
  well-constructed visualizations, graphs and tables.
\item
  Comparisons of the population mean difference for at least one
  quantitative outcome across a set of two (or three) groups, including
  appropriate demonstrations of the reasons behind the choices you made
  between parametric, non-parametric and bootstrap procedures.
\item
  Comparisons of the population proportions for at least one categorical
  outcome across your set of two (or three) groups, including
  appropriately interpreted point estimates and confidence intervals.
\end{itemize}

Note well that Study 1 is \textbf{not} about building sophisticated
statistical models, and using them to make predictions. That's Study 2.

\hypertarget{study-2-is-about-building-a-model-and-making-predictions.}{%
\section{Study 2 is about building a model, and making
predictions.}\label{study-2-is-about-building-a-model-and-making-predictions.}}

\textbf{Study 2} involves data about a \textbf{research question that
you will propose}, involving data of interest to you. Thus, everyone
will be working with a different data set. You will complete all
elements of a data science project designed to create a statistical
model for a \emph{quantitative} outcome, then use it for prediction, and
assess the quality of those predictions.

\begin{quote}
All models are wrong but some are useful.
\href{https://en.wikipedia.org/wiki/All_models_are_wrong}{George E. P.
Box}
\end{quote}

In Study 2, you will be building a multiple linear regression model, and
using it to predict a quantitative outcome of interest. The tools
necessary for Study 2 appear in each Part of the course, especially Part
C, and include the following\ldots{}

\begin{itemize}
\tightlist
\item
  Describing the experimental or observational study design used to
  gather the data, as well as the complete data collection process.
\item
  Sharing the complete raw data in an appropriate way with a
  statistician (Dr.~Love). This means that, in general, data including
  protected health information are \emph{not} appropriate for this
  project.
\item
  Developing appropriate research questions that lead to the
  identification of smart measures for predictors and outcomes, and then
  the development of a prediction model using multiple linear
  regression.
\item
  Using a training sample to develop a model, and present the process
  that leads to a final set of 2-3 candidate models in the training
  sample.
\item
  Using a test sample to evaluate the quality of predictions from each
  of the candidate models, and making a final selection.
\item
  Evaluating the adherence of the data you've collected to the
  assumptions of multiple linear regression, and iterating through the
  model-building process as necessary until the final model shows no
  strong violations of those assumptions.
\end{itemize}

\hypertarget{why-two-studies}{%
\section{Why Two Studies?}\label{why-two-studies}}

The main reason is that I can't figure out a way to get you to think
about all of the things I hope you'll learn from this project in a
single study.

\begin{enumerate}
\def\labelenumi{\arabic{enumi}.}
\tightlist
\item
  I set different tasks for Study 1 and for Study 2, allowing us to
  touch on a wider fraction of the things I hope you'll learn in 431.
\item
  I want some of the work to be done as a class, some in groups, some as
  individuals.
\item
  Some of you have easy access to great data you want to study in this
  class, and in fact, that's a primary motivation for taking the class.
  But not all of you.
\item
  I have to evaluate each of your projects, and there are many students
  in the class. Knowing at least one of the data sets you'll be working
  with helps me manage this.
\item
  Having a broad range of activities to evaluate helps reduce the cost
  of a mistake on any one of them, so that we can build on what you do
  well.
\item
  All of Study 1 can be done by the middle of November, leaving the last
  few weeks of the semester for you to focus on Study 2.
\end{enumerate}

\hypertarget{educational-objectives}{%
\section{Educational Objectives}\label{educational-objectives}}

\begin{quote}
``Statistics has no reason for existence except as the catalyst for
investigation and discovery.''
\href{https://en.wikipedia.org/wiki/George_E._P._Box}{George E. P. Box}
\end{quote}

I am primarily interested in your learning something interesting, useful
and even valuable from your project. An effective project will
demonstrate:

\begin{enumerate}
\def\labelenumi{\arabic{enumi}.}
\tightlist
\item
  The ability to create and formulate research questions that are
  statistically and scientifically appropriate.
\item
  The ability to turn research questions into measures of interest.
\item
  The ability to pull and merge and clean and tidy data, then present
  the data set following
  \href{https://github.com/jtleek/datasharing}{Jeff Leek's guide to
  sharing data with a statistician}.
\item
  The ability to identify appropriate estimation / testing procedures
  for the class survey using both continuous and categorical outcomes.
\item
  The ability to build a reasonable model, including interactions and
  transformations to deal with non-linearity, assess the quality of the
  model and residual plots, then use the model to make predictions.
\item
  The ability to build a Table 1 to showcase potential differences
  between variables.
\item
  The ability to identify and (with help) solve problems that crop up
\item
  The ability to comment on your work within code, and in written and
  oral presentation.
\item
  The ability to build a Markdown-based report and a Markdown-based set
  of slides for presentation.
\end{enumerate}

\hypertarget{taskA}{%
\chapter{Task A (The Proposal) Instructions}\label{taskA}}

Task A requires you to accomplish the following:

On 2018-09-25, you will become part of a \textbf{group} of about five
people, and your group will:

\begin{enumerate}
\def\labelenumi{\arabic{enumi}.}
\tightlist
\item
  develop and propose 2-3 ``research questions'' for Study 1 (The Class
  Survey). Here, your research questions must clearly identify
  meaningful statistical comparisons.
\item
  propose 6-10 ``homemade'' survey questions for Study 1 that relate to
  your research questions, and
\item
  propose a ``scale'' for Study 1 that also relates to your research
  questions.
\end{enumerate}

As an \textbf{individual}, you will also:

\begin{enumerate}
\def\labelenumi{\arabic{enumi}.}
\setcounter{enumi}{3}
\tightlist
\item
  develop and propose a meaningful summary of your ideas and research
  question for Study 2 (Your Data). Your research question needs to
  clearly relate to modeling and prediction of a quantitative outcome on
  the basis of a set of predictor variables.
\item
  identify and present a detailed description of a data set that is
  likely to lead to an answer to the research question proposed for
  Study 2, and that is appropriate for use in this project.
\end{enumerate}

The rest of this section contains guidance as to what sort of questions
your group will need to propose for the class survey in Study 1, and as
to what sorts of data sets and research questions are appropriate for
your Study 2 proposal.

\hypertarget{deadline-and-submission-information}{%
\section{Deadline and Submission
information}\label{deadline-and-submission-information}}

The project groups will only apply to Project Tasks A, C and D. You will
become part of a project group on 2018-09-25 and the groups will disband
after the survey is finalized in late October.

Task A is due at noon on 2018-10-12. You will submit your Task A work
through \href{https://canvas.case.edu/}{Canvas}.

\begin{itemize}
\tightlist
\item
  The group work (Parts 1-3) need to be submitted by one of the members
  of your group.

  \begin{itemize}
  \tightlist
  \item
    If you are not the person submitting this information for your
    group, then your Task A submission should begin with the statement:
    ``Parts 1-3 of Task A were submitted for my group by PERSON'S
    NAME.''
  \end{itemize}
\item
  All students need to submit Parts 4 and 5 of Task A, individually.
\item
  Please note that \textbf{Task A and Task B} are due at the same time.
\end{itemize}

\hypertarget{study-1-work-for-task-a}{%
\section{Study 1 work for Task A}\label{study-1-work-for-task-a}}

In Study 1, you will survey your fellow students through an online
instrument (containing somewhere around 100 items) that we will develop
in Tasks A, C and D in September and October and then administer in the
final week of October (Task E).

Students in the class will develop the items for this instrument in 10
groups of about 5 people per group. The final survey will include
questions generated by each of the 10 groups in the class.

The course survey will be done online, and the respondents
(de-identified, of course) will include all students in the current 431
class, plus the teaching assistants, and perhaps some volunteers from
previous iterations of the course, in an effort to get a reasonable (but
by no means random or representative) sample of graduate students at
CWRU.

Remember that Study 1 is about making comparisons between groups.

\hypertarget{research-questions}{%
\subsection{Research Questions}\label{research-questions}}

The first part of Task A requires your group to develop and propose 2-3
``research questions'' for Study 1 (The Class Survey).

\begin{quote}
A research question is the fundamental core of a research project,
study, or review of literature. It focuses the study, determines the
methodology, and guides all stages of inquiry, analysis, and reporting.
\href{https://researchrundowns.com/intro/writing-research-questions/}{Source}
\end{quote}

\begin{itemize}
\tightlist
\item
  The research questions your project group will prepare for Study 1
  should state the study objective in terms that allow us to apply
  statistical methods to test data to obtain an answer.
\item
  Each research question should be written in the form of a comparison
  of 2-3 exposures or groups in terms of an outcome.
\item
  At least one of your research questions needs to compare groups on a
  quantitative outcome, and at least one needs to compare groups on a
  categorical outcome (containing no more than 5 categories.)
\end{itemize}

Quoting Roger Peng, from
\href{https://bookdown.org/rdpeng/exdata/}{Exploratory Data Analysis
with R}:

\begin{quote}
Formulating a question can be a useful way to guide the exploratory data
analysis process and to limit the exponential number of paths that can
be taken with any sizeable dataset. In particular, a sharp question or
hypothesis can serve as a dimension reduction tool that can eliminate
variables that are not immediately relevant to the question. For
example, (suppose that we are interested in) looking at an air pollution
dataset from the U.S. Environmental Protection Agency (EPA).
\end{quote}

\begin{quote}
A general question one could ask is ``Are air pollution levels higher on
the east coast than on the west coast?'' But a more specific question
might be ``Are hourly ozone levels on average higher in New York City
than they are in Los Angeles?''
\end{quote}

\begin{quote}
Note that both questions may be of interest, and neither is right or
wrong. But the first question requires looking at all pollutants across
the entire east and west coasts, while the second question only requires
looking at single pollutant in two cities. It's usually a good idea to
spend a few minutes to figure out what is the question you're really
interested in, and narrow it down to be as specific as possible (without
becoming uninteresting).
\end{quote}

\hypertarget{checklist-for-research-questions}{%
\subsubsection{Checklist for Research
Questions}\label{checklist-for-research-questions}}

\begin{enumerate}
\def\labelenumi{\arabic{enumi}.}
\tightlist
\item
  Is our research question (RQ) something that we are curious about and
  that others might care about?
\item
  Does our RQ present an issue on which we can justify a stand prior to
  data collection about what we think will happen?
\item
  Is our RQ too broad, too narrow, or OK, given the time frame and
  restrictions of this survey?
\item
  Is our RQ measurable? What type of information do we need? Can I find
  a way to ask a limited number of survey questions in such a way to
  allow me to (after the data are collected) either support or
  contradict a position on my RQ?
\end{enumerate}

\begin{itemize}
\tightlist
\item
  Adapted from
  \href{http://www8.esc.edu/htmlpages/writerold/menus.htm\#develop}{this
  online tutorial from Empire State College}
\end{itemize}

\hypertarget{tips-on-writing-good-research-questions}{%
\subsubsection{Tips on Writing Good Research
Questions}\label{tips-on-writing-good-research-questions}}

\begin{itemize}
\tightlist
\item
  \href{https://sites.duke.edu/urgws/files/2014/02/Research-Questions_WS-handout.pdf}{Duke}
  has a nice overview online of key issues.
\item
  \href{https://www.vanderbilt.edu/writing/wp-content/uploads/sites/164/2016/10/Formulating-Your-Research-Question.pdf}{Vanderbilt}
  has some nice materials, built from the
  \href{http://www8.esc.edu/htmlpages/writerold/menus.htm\#develop}{tutorial
  at Empire State College} quoted earlier
\item
  Jeff Leek provides several relevant tips in \emph{The Elements of Data
  Analytic Style}
\item
  \url{https://researchrundowns.com/intro/writing-research-questions/}
  has some excellent tips on wording
\end{itemize}

\hypertarget{specifying-survey-questions}{%
\subsection{Specifying Survey
Questions}\label{specifying-survey-questions}}

The second part of Task A requires your group to develop and propose
6-10 ``homemade'' survey questions for Study 1 that relate to your
research questions.

Your group will need to specify the exact wording for your survey
questions (and potential answers for any categorical responses.) This
will likely require some editing and rework, once we have the complete
set of proposed questions from all students. Be prepared to revise and
resubmit, quickly, so that all items can be resolved in time for
publication of the draft survey.

Of your 6-10 survey questions \ldots{}

\begin{itemize}
\tightlist
\item
  at least two should ask the respondents to provide you with a
  \textbf{number} that expresses a quantitative outcome of interest to
  you, and these outcomes should relate closely to at least one of your
  research questions.

  \begin{itemize}
  \tightlist
  \item
    If you are asking people to respond to a prompt using a rating, that
    rating should be expressed on a wide scale. Our preference is a
    0-100 scale for quantitative items, where 0 represents the most
    negative reaction and 100 the most positive reaction to the item.
  \item
    One common choice is to make a statement and ask for agreement with
    that statement on a scale from 0 = Strongly Disagree to 100 =
    Strongly Agree.
  \item
    The reason we prefer a 0-100 scale is to increase variation in our
    responses, as compared to, say, a 1-10 scale.
  \item
    When responding to items using a scale like this on the actual
    survey, please use the whole scale.
  \end{itemize}
\item
  \textbf{at least two} should ask the respondents to provide you with a
  response that expresses a categorical outcome of interest to you, and
  these, too, should relate to at least one of your research questions.

  \begin{itemize}
  \tightlist
  \item
    You will need to specify each of the available responses that you
    wish to use in the survey.
  \item
    No more than five options for your categorical outcome, please.
  \item
    The response options you specify should be mutually exclusive and
    collectively exhaustive.
  \end{itemize}
\item
  \textbf{at least two} should ask the respondents to categorize
  themselves into one of two (or three) groups.

  \begin{itemize}
  \tightlist
  \item
    Be aware that you will need to have at least 10 students in each
    group in order to build a semi-reasonable analysis.
  \item
    You should expect that 50-55 people will actually respond to the
    survey, in total.
  \item
    Again, these groupings should be linked to your research questions.
  \end{itemize}
\end{itemize}

You are welcome to submit exactly 6, or as many as 10 total survey
questions in this part of the Task. It is likely that some of your
survey questions will not correspond to some of your research questions,
and that's OK, but each survey question should be linked to at least one
of your research questions.

\hypertarget{dr.love-will-include-15-survey-questions-automatically}{%
\subsubsection{Dr.~Love will include 15 Survey Questions
Automatically}\label{dr.love-will-include-15-survey-questions-automatically}}

The following items will be included in the survey to identify
``groups'' of students in a reasonable way. As a result, you should not
ask these questions in your proposed list, although you can and should
consider whether these groupings would be good candidates for
application to your research questions.

The following 7 items will have yes/no responses, and thus produce
binary groups for analysis.

\begin{enumerate}
\def\labelenumi{\arabic{enumi}.}
\tightlist
\item
  Were you born in the United States?
\item
  Is English the language you speak better than any other?
\item
  Do you identify as female?
\item
  Do you wear prescription glasses or contact lenses?
\item
  Before taking 431, had you ever used R before?
\item
  Are you currently married or in a stable domestic relationship?
\item
  Have you smoked 100 cigarettes or more in your entire life?
\end{enumerate}

The next eight items generate non-binary responses. Together, after the
survey is complete, we will identify ``cutpoints'' for these eight items
to identify groups of meaningful size.

\begin{enumerate}
\def\labelenumi{\arabic{enumi}.}
\setcounter{enumi}{7}
\tightlist
\item
  In what year were you born?
\item
  How would you rate your current health overall (Excellent, Very Good,
  Good, Fair, Poor)
\item
  For how long, in months, have you lived in Northeast Ohio?
\item
  What is your height in inches? (If you are five feet, eight inches
  tall, please write 68 inches. To convert from centimeters to inches,
  multiply your height in centimeters by 0.3937, and then round the
  result to the nearest inch.)
\item
  What is your weight in pounds? (To convert from kilograms to pounds,
  multiply your weight in kilograms by 2.2046, and then round the result
  to the nearest pound.)
\item
  What is your pulse rate, in beats per minute? (Please either use a
  tracking device, or count your pulse for 15 seconds then multiply by
  4)
\item
  Last week, on how many days did you exercise? (0 - 7)
\item
  Last night, how many hours of sleep did you get?
\end{enumerate}

\hypertarget{permitted-types-of-items}{%
\subsubsection{Permitted Types of
Items}\label{permitted-types-of-items}}

The survey will be conducted using a Google Form, rather than Survey
Monkey or some other tool. Thus, we have a somewhat restricted set of
item types.

For \textbf{quantitative measures}, Google Forms permit the use of

\begin{enumerate}
\def\labelenumi{\arabic{enumi}.}
\tightlist
\item
  a \emph{short answer} item without any restrictions on the response
  (except a character limit)
\item
  a \emph{short answer} item where respondents are forced to insert a
  number within a given range through a validation process that only
  accepts the response if it falls within the specified limits.
\item
  \emph{linear scale} items for ordered categorical ratings (but only on
  a scale of up to ten points - i.e.~1 to 10)
\end{enumerate}

For \textbf{categorical measures}, Google Forms permit the use of

\begin{enumerate}
\def\labelenumi{\arabic{enumi}.}
\tightlist
\item
  \emph{multiple choice} items for endorsing a single choice from a
  group of 2-10 alternatives.
\item
  \emph{checkbox} items for the endorsement of one or more choices from
  a group of 2-10 alternatives.
\item
  \emph{linear scale} items for ordered categorical ratings (often on a
  1-X scale, where X is between 2 and 10)
\item
  \emph{dropdown} items for selections of one option from a group of 2-3
  choices.
\end{enumerate}

\hypertarget{old-class-surveys}{%
\subsubsection{Old Class Surveys}\label{old-class-surveys}}

The surveys (each containing at least 100 items) from
\href{https://github.com/THOMASELOVE/431-2018-project/blob/master/oldsurveys/2014_431_class_survey.pdf}{2014},
\href{https://github.com/THOMASELOVE/431-2018-project/blob/master/oldsurveys/2015_431_class_survey.pdf}{2015},
\href{https://github.com/THOMASELOVE/431-2018-project/blob/master/oldsurveys/2016_431_class_survey.pdf}{2016}
and
\href{https://github.com/THOMASELOVE/431-2018-project/blob/master/oldsurveys/2017_431_class_survey.pdf}{2017}
are available as PDF documents on our web site.

Remember that the rules used this year have been modified from what has
been used for the project previously.

\hypertarget{specifying-a-scale}{%
\subsection{Specifying a ``Scale''}\label{specifying-a-scale}}

The third part of Task A requires your group to identify and propose a
``scale'' of items for Study 1 (The Class Survey).

Your group needs to specify a published scale (available for free public
use) to generate an outcome or grouping(s) of interest from those
completing the survey. You will have to provide a complete reference to
the scale (online, ideally) and specify each of the items in the scale,
and how the scale is then evaluated, in all necessary detail to allow us
to review and replicate the scale in practice.

The word ``scale'' is used in many different ways. In this case, by a
scale I mean a published list of items, usually accompanied by a scoring
rubric that provides some sort of composite score or scores. Examples of
scales we have used in the past include:

\begin{itemize}
\tightlist
\item
  Two \href{https://chirr.nlm.nih.gov/health-orientation.php}{Health
  Consciousness Scales}, one due to Gould another to Dutta-Bregman
  \href{https://chirr.nlm.nih.gov/health-orientation.php}{Gould Health
  Consciousness Scale}
\item
  The
  \href{https://gosling.psy.utexas.edu/scales-weve-developed/ten-item-personality-measure-tipi/ten-item-personality-inventory-tipi/}{Ten-Item
  Personality Inventory}
\item
  The
  \href{https://das.nh.gov/wellness/Docs/Percieved\%20Stress\%20Scale.pdf}{Perceived
  Stress Scale}
\item
  The
  \href{https://www.sleepapnea.org/assets/files/pdf/ESS\%20PDF\%201990-97.pdf}{Epworth
  Sleepiness Scale}
\end{itemize}

Your group will need to verify explicitly in your Task A materials that
the scale your group proposes is freely available for use by anyone,
without any fees or registration requirements.

\hypertarget{study-2-work-for-task-a}{%
\section{Study 2 Work for Task A}\label{study-2-work-for-task-a}}

You, individually, will present a proposal \textbf{summary} (\textless{}
300 words) and a brief \textbf{data description} for Study 2 in Task A.

You will be building a multiple linear regression model, and using it to
predict your outcome of interest.

\hypertarget{the-proposal-summary}{%
\subsection{The Proposal Summary}\label{the-proposal-summary}}

The fourth part of Task A is to develop and propose a meaningful summary
of your ideas and research question for Study 2 (Your Data).

Your summary should begin with a title for your Study 2. Take the time
to come up with a good, interesting title. You are going to work hard on
this thing; please resist the temptation to kill my interest at the
start by calling it ``431 Statistics Project'' or anything else that
shows a similar lack of effort.

Provide me a very brief summary of what you're trying to accomplish -
specifically, what your research question is, and what you hypothesize
will happen.

The five most important things to do in the summary are:

\begin{enumerate}
\def\labelenumi{\arabic{enumi}.}
\tightlist
\item
  Write clearly. My best advice is to finish the summary as soon as you
  can, and then give it to someone else to read, who can criticize it
  for lack of clarity in the writing.
\item
  Specify the topic of interest, and motivate your study of it.
\item
  Explicitly specify your key research question, which should be stated
  as a question, and which should clearly and naturally lead to a
  prediction model for a quantitative outcome.
\item
  Explain what you hypothesize will happen, and
\item
  Explicitly link your key research question to the data set you
  describe in your data description.
\end{enumerate}

The summary is the heart of the proposal, and requires some care. You
will need to convince me that your topic is interesting, your data are
relevant, and building a model and making predictions of a quantitative
outcome using the predictors available to you will be worthwhile.

\begin{itemize}
\tightlist
\item
  The summary ends with a statement of the research question or
  questions (you may have one, or possibly two.) An excellent question
  conveys the main objective of the study in terms that allow us to
  apply statistical models to describe an association between one or
  more predictors and a quantitative outcome. Some advice on writing a
  good research question is provided below, after the data set
  description information.
\item
  It should be possible for me to explain your study accurately just by
  reading this summary. If it's not possible, it will come back to you
  for a REDO. Statistics is a details business. Get the details right.
\item
  This summary should be less than 300 words.
\item
  Use complete English sentences. Write in plain language. Use words we
  all know. Avoid jargon. And
  \href{https://thomaselove.github.io/2018-431-syllabus/a-few-general-writingpresenting-tips.html}{look
  at the general suggestions about writing in the Course Syllabus}.
\end{itemize}

\hypertarget{the-data-description}{%
\subsection{The Data Description}\label{the-data-description}}

Your data description can be as long as it needs to be, although two
pages is usually more than enough. It should include:

\begin{enumerate}
\def\labelenumi{\arabic{enumi}.}
\item
  Your data source, which can be an online source (in which case include
  a working link), a published paper or journal article (in which case I
  need a link and a PDF copy of the paper), or unpublished data (in
  which case I need the details of how the data were gathered).
\item
  A thorough description of the data collection process, with complete
  details as to the nature of the variables, the setting for data
  collection, and complete details of any apparatus you used which may
  affect results.
\item
  Specification of the people and methods involved.

  \begin{itemize}
  \tightlist
  \item
    Who are the subjects under study?
  \item
    When were the data gathered? By whom?
  \item
    How many subjects are included?
  \item
    What caused subjects to be included or excluded from the study?
  \end{itemize}
\item
  Your planned \textbf{quantitative} outcome, which must relate directly
  to the research question you specified above. Provide a complete
  definition, including specifying the exact wording of the question or
  details of the measurement procedure used to obtain the outcome. If
  available, you can also include descriptions of secondary
  \textbf{quantitative} outcomes. Your outcomes must be quantitative in
  Study 2.
\item
  Your predictors of interest, which should also relate to the research
  question in an obvious way. Again, define the variables carefully, as
  you did with the outcome.
\item
  If you already have the data, tell me that. If you don't, specify any
  steps you must still take in order to get the data, and specify the
  date by which you will have your data (must be no later than November
  1.)
\end{enumerate}

\hypertarget{data-restrictions}{%
\subsection{Data Restrictions}\label{data-restrictions}}

Study 2 data sets MUST

\begin{itemize}
\tightlist
\item
  contain between 250 and 250,000 distinct observations,
\item
  contain at least one quantitative outcome variable,
\item
  contain at least four predictor variables, one of which may be
  identified as the ``key'' predictor of interest,
\item
  include at least one quantitative predictor variable, and at least one
  categorical predictor variable,
\item
  include a complete description of how the data were gathered, so that
  information must be publicly available,
\item
  be in your hands no later than November 1,
\item
  be shared data with a statistician (Dr.~Love) following
  \href{https://github.com/jtleek/datasharing}{Jeff Leek's guide to
  sharing data with a statistician} as part of Task F. This means you
  need to have access to the data in the raw, and it means that I have
  to be able to have access to it in the raw, as well. - be capable of
  being fully cited for any and all data elements, including a complete
  codebook, as this must be provided as part of your proposal.
\end{itemize}

While there are some great resources available to some people in this
class by virtue of their affiliation with one of the health systems in
town, I can do nothing to get you access to health system specific data
as part of your project for this class or for 432, and in general, data
from those sources are not especially appropriate because of issues with
protected health information.

No more than two students in the class can work on the same data. If two
of you have data you would each like to work on, that may be OK, but
you'll need to generate separate research questions and perform your
analyses and the Project Tasks separately.

I am not interested in you using pre-cleaned data from an educational
repository, such as:

\begin{itemize}
\tightlist
\item
  \href{http://www.lerner.ccf.org/qhs/datasets/}{this one at the
  Cleveland Clinic}, or
  \href{http://biostat.mc.vanderbilt.edu/wiki/Main/DataSets}{this one at
  Vanderbilt University}, or
  \href{http://www.stat.ucla.edu/projects/datasets/}{this one at UCLA},
  or \href{http://www.stat.ufl.edu/~winner/datasets.html}{this one at
  the University of Florida}, or
  \href{http://people.sc.fsu.edu/~jburkardt/datasets/datasets.html}{this
  one at Florida State University}, or
\item
  \href{http://lib.stat.cmu.edu/datasets/}{StatLib at Carnegie-Mellon
  University}, or
  \href{http://www.amstat.org/publications/jse/jse_data_archive.htm}{the
  Journal of Statistics Education Data Archive}, or
\item
  the data sets gathered in the fivethirtyeight package, the mosaic
  package, the cars package, the datasets package, or any other R
  package designed primary for teaching, or
\item
  \href{http://www.statsci.org/datasets.html}{StatSci.org's repository
  of textbook examples and ready for teaching data}, or
\item
  any of the many textbook-linked repositories of data sets, like
  \href{http://www.lock5stat.com/datapage.html}{this one for Statistics:
  Unlocking the Power of Data}, or
\item
  any similar repository Professor Love deems to be inappropriate
\end{itemize}

\hypertarget{some-potentially-useful-data-sources}{%
\subsection{Some Potentially Useful Data
Sources}\label{some-potentially-useful-data-sources}}

The ideal choice of data source for this project is a public-use version
of a meaningful data set without access restrictions. With so many
students in the class, I cannot be responsible for supervising your work
with restricted data personally. Some appealing sources to explore
include:

\begin{itemize}
\tightlist
\item
  the new \href{https://toolbox.google.com/datasetsearch}{Google
  Datasets Search}
\item
  \url{https://www.data.gov/} The home of the U.S. Government's open
  data
\item
  \url{http://www.census.gov/data.html} The U.S. Census Bureau has many
  interesting data sets, including the
  \href{http://www.census.gov/programs-surveys/cps.html}{Current
  Population Survey}
\item
  \url{http://www.healthdata.gov/} 125 years of U.S. Health Care Data
\item
  \url{http://www.cdc.gov/nchs/nhanes/index.htm} National Health and
  Nutrition Examination Survey.

  \begin{itemize}
  \tightlist
  \item
    You may want to look at
    \href{https://cran.r-project.org/web/packages/nhanesA/vignettes/Introducing_nhanesA.html}{the
    nhanesA package in R}
  \end{itemize}
\item
  \url{http://dashboard.healthit.gov/datadashboard/data.php} Office of
  the National Coordinator for Health IT's dashboard
\item
  \url{http://www.icpsr.umich.edu/icpsrweb/} ICSPR (Inter-university
  Consortium for Political and Social Research) is a source for many
  public-use data sets

  \begin{itemize}
  \tightlist
  \item
    This includes the
    \href{http://www.icpsr.umich.edu/icpsrweb/HMCA/}{Health and Medical
    Care data archive of the Robert Wood Johnson Foundation}
  \end{itemize}
\item
  \url{http://gss.norc.org/} The General Social Survey
\item
  \url{http://www.bls.gov/data/} Bureau of Labor Statistics
\item
  \url{http://nces.ed.gov/surveys/} National Center for Education
  Statistics
\item
  \url{http://www.odh.ohio.gov/healthstats/dataandstats.aspx} Ohio
  Department of Health
\item
  \url{http://open.canada.ca/en} Canada Open Data
\item
  \url{http://digital.nhs.uk/home} Health data sets from the UK National
  Health Service.
\item
  \url{http://www.who.int/en/} World Health Organization
\item
  \url{http://www.unicef.org/statistics/} UNICEF has some available data
  on women and children
\item
  \url{http://www.pewinternet.org/datasets/} Pew Research Center's
  Internet Project
\item
  \url{http://portals.broadinstitute.org/cgi-bin/cancer/datasets.cgi}
  Broad Institute's Cancer Program
\item
  \url{http://www.kdnuggets.com/datasets/index.html} is a big index of
  lots of available data repositories
\item
  \url{https://www.kaggle.com/} Kaggle competition data sets are
  attractive to students occasionally, but I've seen a lot of them
  before and don't really want to see them again.
\end{itemize}

I cannot guarantee the quality of any of the data sets available at
these sites, but I've spent at least a little time at many of them in
recent months.

\hypertarget{if-you-are-storing-your-own-data}{%
\subsection{If you are storing your own
data}\label{if-you-are-storing-your-own-data}}

An extremely useful link for those of you \textbf{building a spreadsheet
to store data} is \href{http://kbroman.org/dataorg/}{Karl Broman's
tutorial} on the subject. No one was born knowing this stuff - take a
look.

\hypertarget{evaluating-task-a}{%
\section{Evaluating Task A}\label{evaluating-task-a}}

Dr.~Love will evaluate all proposals (Task A) personally, in the order
in which they are received. Proposals will receive one of two grades: OK
or REDO. That grade will be posted to Canvas. REDO will be accompanied
with specific requests in the form of a Canvas comment that should be
accomplished within a short time window (approximately 24 hours). If you
materially deviate from these specifications, Dr.~Love will not evaluate
your proposal other than to re-specify what needs to be fixed before he
will respond.

\begin{itemize}
\tightlist
\item
  A score of OK is worth 10/10 points for Task A, once Task B is also
  complete.
\item
  You (and/or your group, if there are problems with parts 1-3) must
  REDO the Proposal until you reach OK. Sometimes, that's more than
  once.
\end{itemize}

\hypertarget{taskB}{%
\chapter{Task B (Presentation Sign-Up) Instructions}\label{taskB}}

\hypertarget{deadline-and-submission-information-1}{%
\section{Deadline and Submission
information}\label{deadline-and-submission-information-1}}

Task B is due at noon on 2018-10-12. Submit your Task B work by
completing the Google Form linked at
\url{http://bit.ly/431-2018-project-signup-taskB}.

\begin{itemize}
\item
  Please note that Task A is also due at the same time.
\item
  All students must specify a minimum of 8 time slots, on at least two
  different days, when they can give their presentation.
\item
  You will also be able to specify your two favorite time slots among
  those you have chosen.
\item
  The presentation dates are 2018-12-10, 2018-12-11 and 2018-12-13.

  \begin{itemize}
  \tightlist
  \item
    University classes end December 7.
  \item
    December 10 is one of the official University Reading Days, and
    December 11 and 13 are Final Exam Days.
  \end{itemize}
\item
  If you have some special problem or concern or need to give your
  presentation before 2018-12-10, there is a space to tell Dr.~Love
  about that at the end of the form.
\end{itemize}

\hypertarget{available-time-slots}{%
\section{Available Time Slots}\label{available-time-slots}}

There are 55 available time slots, listed below.

\hypertarget{monday-2018-12-10-morning}{%
\subsection{Monday 2018-12-10 morning}\label{monday-2018-12-10-morning}}

\begin{longtable}[]{@{}rrccc@{}}
\toprule
Time Slot & Date & Start & Finish & \textbf{Arrival Time}\tabularnewline
\midrule
\endhead
1 & MON 12-10 & 8:00 AM & 8:20 AM & \textbf{7:50 AM}\tabularnewline
2 & MON 12-10 & 8:25 AM & 8:45 AM & \textbf{8:15 AM}\tabularnewline
3 & MON 12-10 & 8:50 AM & 9:10 AM & \textbf{8:40 AM}\tabularnewline
4 & MON 12-10 & 9:15 AM & 9:35 AM & \textbf{9:05 AM}\tabularnewline
5 & MON 12-10 & 9:40 AM & 10:00 AM & \textbf{9:30 AM}\tabularnewline
6 & MON 12-10 & 10:15 AM & 10:35 AM & \textbf{10:05 AM}\tabularnewline
7 & MON 12-10 & 10:40 AM & 11:00 AM & \textbf{10:30 AM}\tabularnewline
8 & MON 12-10 & 11:05 AM & 11:25 AM & \textbf{10:55 AM}\tabularnewline
9 & MON 12-10 & 11:30 AM & 11:50 AM & \textbf{11:20 AM}\tabularnewline
\bottomrule
\end{longtable}

\hypertarget{monday-2018-12-10-afternoon}{%
\subsection{Monday 2018-12-10
afternoon}\label{monday-2018-12-10-afternoon}}

\begin{longtable}[]{@{}rrccc@{}}
\toprule
Time Slot & Date & Start & Finish & \textbf{Arrival Time}\tabularnewline
\midrule
\endhead
10 & MON 12-10 & 12:35 PM & 12:55 PM & \textbf{12:25 PM}\tabularnewline
11 & MON 12-10 & 1:00 PM & 1:20 PM & \textbf{12:50 PM}\tabularnewline
12 & MON 12-10 & 1:25 PM & 1:45 PM & \textbf{1:15 PM}\tabularnewline
13 & MON 12-10 & 1:50 PM & 2:10 PM & \textbf{1:40 PM}\tabularnewline
14 & MON 12-10 & 2:15 PM & 2:35 PM & \textbf{2:05 PM}\tabularnewline
15 & MON 12-10 & 2:40 PM & 3:00 PM & \textbf{2:30 PM}\tabularnewline
16 & MON 12-10 & 3:15 PM & 3:35 PM & \textbf{3:05 PM}\tabularnewline
17 & MON 12-10 & 3:40 PM & 4:00 PM & \textbf{3:30 PM}\tabularnewline
18 & MON 12-10 & 4:05 PM & 4:25 PM & \textbf{3:55 PM}\tabularnewline
19 & MON 12-10 & 4:30 PM & 4:50 PM & \textbf{4:20 PM}\tabularnewline
20 & MON 12-10 & 4:55 PM & 5:15 PM & \textbf{4:45 PM}\tabularnewline
\bottomrule
\end{longtable}

\hypertarget{tuesday-2018-12-11-morning}{%
\subsection{Tuesday 2018-12-11
morning}\label{tuesday-2018-12-11-morning}}

\begin{longtable}[]{@{}rrccc@{}}
\toprule
Time Slot & Date & Start & Finish & \textbf{Arrival Time}\tabularnewline
\midrule
\endhead
21 & TUE 12-11 & 8:00 AM & 8:20 AM & \textbf{7:50 AM}\tabularnewline
22 & TUE 12-11 & 8:25 AM & 8:45 AM & \textbf{8:15 AM}\tabularnewline
23 & TUE 12-11 & 8:50 AM & 9:10 AM & \textbf{8:40 AM}\tabularnewline
24 & TUE 12-11 & 9:15 AM & 9:35 AM & \textbf{9:05 AM}\tabularnewline
25 & TUE 12-11 & 9:40 AM & 10:00 AM & \textbf{9:30 AM}\tabularnewline
26 & TUE 12-11 & 10:15 AM & 10:35 AM & \textbf{10:05 AM}\tabularnewline
27 & TUE 12-11 & 10:40 AM & 11:00 AM & \textbf{10:30 AM}\tabularnewline
28 & TUE 12-11 & 11:05 AM & 11:25 AM & \textbf{10:55 AM}\tabularnewline
29 & TUE 12-11 & 11:30 AM & 11:50 AM & \textbf{11:20 AM}\tabularnewline
\bottomrule
\end{longtable}

\hypertarget{tuesday-2018-12-11-afternoon}{%
\subsection{Tuesday 2018-12-11
afternoon}\label{tuesday-2018-12-11-afternoon}}

\begin{longtable}[]{@{}rrccc@{}}
\toprule
Time Slot & Date & Start & Finish & \textbf{Arrival Time}\tabularnewline
\midrule
\endhead
30 & TUE 12-11 & 12:35 PM & 12:55 PM & \textbf{12:25 PM}\tabularnewline
31 & TUE 12-11 & 1:00 PM & 1:20 PM & \textbf{12:50 PM}\tabularnewline
32 & TUE 12-11 & 1:25 PM & 1:45 PM & \textbf{1:15 PM}\tabularnewline
33 & TUE 12-11 & 1:50 PM & 2:10 PM & \textbf{1:40 PM}\tabularnewline
34 & TUE 12-11 & 2:15 PM & 2:35 PM & \textbf{2:05 PM}\tabularnewline
35 & TUE 12-11 & 2:40 PM & 3:00 PM & \textbf{2:30 PM}\tabularnewline
36 & TUE 12-11 & 3:15 PM & 3:35 PM & \textbf{3:05 PM}\tabularnewline
\bottomrule
\end{longtable}

\hypertarget{thursday-2018-12-13-morning}{%
\subsection{Thursday 2018-12-13
morning}\label{thursday-2018-12-13-morning}}

\begin{longtable}[]{@{}rrccc@{}}
\toprule
Time Slot & Date & Start & Finish & \textbf{Arrival Time}\tabularnewline
\midrule
\endhead
37 & THU 12-13 & 8:00 AM & 8:20 AM & \textbf{7:50 AM}\tabularnewline
38 & THU 12-13 & 8:25 AM & 8:45 AM & \textbf{8:15 AM}\tabularnewline
39 & THU 12-13 & 8:50 AM & 9:10 AM & \textbf{8:40 AM}\tabularnewline
40 & THU 12-13 & 9:15 AM & 9:35 AM & \textbf{9:05 AM}\tabularnewline
41 & THU 12-13 & 9:40 AM & 10:00 AM & \textbf{9:30 AM}\tabularnewline
42 & THU 12-13 & 10:15 AM & 10:35 AM & \textbf{10:05 AM}\tabularnewline
43 & THU 12-13 & 10:40 AM & 11:00 AM & \textbf{10:30 AM}\tabularnewline
44 & THU 12-13 & 11:05 AM & 11:25 AM & \textbf{10:55 AM}\tabularnewline
45 & THU 12-13 & 11:30 AM & 11:50 AM & \textbf{11:20 AM}\tabularnewline
\bottomrule
\end{longtable}

\hypertarget{thursday-2018-12-13-afternoon}{%
\subsection{Thursday 2018-12-13
afternoon}\label{thursday-2018-12-13-afternoon}}

\begin{longtable}[]{@{}rrccc@{}}
\toprule
Time Slot & Date & Start & Finish & \textbf{Arrival Time}\tabularnewline
\midrule
\endhead
46 & THU 12-13 & 12:35 PM & 12:55 PM & \textbf{12:25 PM}\tabularnewline
47 & THU 12-13 & 1:00 PM & 1:20 PM & \textbf{12:50 PM}\tabularnewline
48 & THU 12-13 & 1:25 PM & 1:45 PM & \textbf{1:15 PM}\tabularnewline
49 & THU 12-13 & 1:50 PM & 2:10 PM & \textbf{1:40 PM}\tabularnewline
50 & THU 12-13 & 2:15 PM & 2:35 PM & \textbf{2:05 PM}\tabularnewline
51 & THU 12-13 & 2:40 PM & 3:00 PM & \textbf{2:30 PM}\tabularnewline
52 & THU 12-13 & 3:15 PM & 3:35 PM & \textbf{3:05 PM}\tabularnewline
53 & THU 12-13 & 3:40 PM & 4:00 PM & \textbf{3:30 PM}\tabularnewline
54 & THU 12-13 & 4:05 PM & 4:25 PM & \textbf{3:55 PM}\tabularnewline
55 & THU 12-13 & 4:30 PM & 4:50 PM & \textbf{4:20 PM}\tabularnewline
\bottomrule
\end{longtable}

\hypertarget{taskC}{%
\chapter{Task C (Survey Editing) Instructions}\label{taskC}}

\hypertarget{deadline-and-submission-information-2}{%
\section{Deadline and Submission
information}\label{deadline-and-submission-information-2}}

Task C is due at noon on 2018-10-23. Submit your Task C work via
\href{https://canvas.case.edu/}{Canvas}. Please note that:

\begin{itemize}
\tightlist
\item
  Task D is also due at the same time.
\item
  We do not have class on 2018-10-23 because of CWRU's Fall Break.
\end{itemize}

Dr.~Love will make the Draft Survey available as soon as possible after
our initial work developing the Survey is complete.

\hypertarget{task-c-has-two-parts.}{%
\section{Task C has two parts.}\label{task-c-has-two-parts.}}

In Task C, you need to submit a single Word document (maximum one page,
12 point font, with your name and Project Task C on the top of the Word
document) containing the following two things:

\hypertarget{a-list-of-corrections-and-clarifications-to-the-existing-items-in-the-draft-survey.}{%
\subsection{A list of corrections and clarifications to the existing
items in the Draft
Survey.}\label{a-list-of-corrections-and-clarifications-to-the-existing-items-in-the-draft-survey.}}

This should include any typographical errors, clarifications or other
edits that you wish to suggest for the items included in the Draft
Survey.

\begin{itemize}
\tightlist
\item
  If you found no errors or items in need of clarification, write a
  sentence saying that.
\item
  If you did find an issue, please be sure to specify the item number
  where you feel a revision is needed.
\end{itemize}

\textbf{An Important Note}: In addition, If you see any items in the
Draft Survey that you, personally, are not comfortable answering, for
whatever reason, \textbf{please indicate that to us} in your response
here, and we will consider revisions appropriately.

\hypertarget{a-list-of-0-3-new-items-that-you-want-us-to-consider-adding-to-the-draft-survey.}{%
\subsection{A list of 0-3 new items that you want us to consider adding
to the Draft
Survey.}\label{a-list-of-0-3-new-items-that-you-want-us-to-consider-adding-to-the-draft-survey.}}

\begin{itemize}
\tightlist
\item
  Note that your new items \emph{can} be but do not \emph{need} to be
  anything you've previously suggested.
\item
  Please begin with the following sentence:
  \texttt{I\ would\ like\ to\ submit\ \#\ new\ items\ for\ consideration.}

  \begin{itemize}
  \tightlist
  \item
    If your number of new items to suggest is zero, then you need not
    write anything else here. - Should you wish to have us include 1-3
    additional items, remember that nothing about sex, drugs, or
    performance in 431 can be asked.
  \end{itemize}
\end{itemize}

For each new item you propose\ldots{}

\begin{enumerate}
\def\labelenumi{\arabic{enumi}.}
\tightlist
\item
  list the complete wording of the new item, being sure to specify the
  type (for instance, short answer, multiple choice, or checkbox) and
  the set of possible responses, as you did in Task A.
\item
  describe (in 2-3 complete sentences per new item) your reasons to
  include the item.

  \begin{itemize}
  \tightlist
  \item
    Good reasons would begin with a statement of what you intend to do.
    As an example of such a statement, consider
    \texttt{I\ wish\ to\ study\ the\ result\ of\ this\ new\ item\ as\ a\ quantitative\ outcome\ across\ groups\ established\ by\ current\ item\ \#***\ from\ the\ survey.}
    Or, perhaps, something like:
    \texttt{I\ wish\ to\ use\ this\ new\ item\ as\ a\ grouping\ variable\ to\ study\ current\ item\ \#***.}
  \item
    In either case, follow your statement with a short explanation as to
    why your new item's result is of interest, and is not already
    captured by the existing survey.
  \end{itemize}
\end{enumerate}

We will not consider more than 3 new items from anyone, and are eager to
hold the total set of new items to 20 or less, across all 51 students. I
hope that data related to each of the accepted project proposals will be
found in the Draft Survey, as that is certainly the intention. So if
that's the case, we are unlikely to need more than a few new items.

\hypertarget{taskD}{%
\chapter{Task D (Survey Comparison Plan) Instructions}\label{taskD}}

\hypertarget{deadline-and-submission-information-3}{%
\section{Deadline and Submission
information}\label{deadline-and-submission-information-3}}

Task D is due at noon on 2018-10-23. Submit your Task D work by
completing the Google Form linked at
\url{http://bit.ly/431-2018-survey-comparison-plan-taskD}.

Please note that:

\begin{itemize}
\tightlist
\item
  Task C is also due at the same time.
\item
  We do not have class on 2018-10-23 because of CWRU's Fall Break.
\end{itemize}

\hypertarget{task-d-requires-you-to-complete-a-google-form}{%
\section{Task D requires you to complete a Google
Form}\label{task-d-requires-you-to-complete-a-google-form}}

The Google Form for Task D is now available at
\url{http://bit.ly/431-2018-survey-comparison-plan-taskD}.

In this form, you will need to specify the list of items from the Draft
Survey that you plan to use in one or more of your six required analyses
for Study 1 (the Class Survey.)

\begin{itemize}
\tightlist
\item
  The Form will ask you to specify by item number and name the items you
  wish to use for each of the six required Analyses for Study 1.
\item
  In addition to the items you select related to each Analysis, you will
  also select two backup quantitative variables, and two backup factors.
\item
  You need not do any analyses connected to the items you originally
  suggested, nor do you need to do analyses that mirror your original
  research questions.
\item
  Items with at least 10 possible responses will be treated as
  quantitative. Other items will be treated as categorical (factors.)

  \begin{itemize}
  \tightlist
  \item
    For ordered categories, you can consider assigning a score to each
    response, then treating that score as quantitative.
  \item
    You are permitted to categorize into a group with 2-4 levels any
    quantitative item you choose.
  \item
    You are permitted to collapse any categories in an item with more
    than 2 categories, as you choose.
  \item
    Some items are part of multiple-item scales. If you want to use a
    scale, specify each item that would go into that scale on the Task D
    form, as applicable.
  \end{itemize}
\end{itemize}

\hypertarget{the-six-required-analyses-for-study-1}{%
\section{The Six Required Analyses for Study
1}\label{the-six-required-analyses-for-study-1}}

\begin{itemize}
\tightlist
\item
  The actual analyses you will need to do include either Analysis 1a or
  1b (but not both), along with Analyses 2-6, as specified below.
\end{itemize}

\begin{longtable}[]{@{}ll@{}}
\toprule
\begin{minipage}[b]{0.42\columnwidth}\raggedright
Analysis\strut
\end{minipage} & \begin{minipage}[b]{0.52\columnwidth}\raggedright
Variables needed\strut
\end{minipage}\tabularnewline
\midrule
\endhead
\begin{minipage}[t]{0.42\columnwidth}\raggedright
{[}1a{]} 2 means via paired samples\strut
\end{minipage} & \begin{minipage}[t]{0.52\columnwidth}\raggedright
Two quantitative (outcomes)\strut
\end{minipage}\tabularnewline
\begin{minipage}[t]{0.42\columnwidth}\raggedright
{[}1b{]} 2 means via independent samples\strut
\end{minipage} & \begin{minipage}[t]{0.52\columnwidth}\raggedright
One quantitative (outcome) and one categorical (2 levels)\strut
\end{minipage}\tabularnewline
\begin{minipage}[t]{0.42\columnwidth}\raggedright
{[}2{]} ANOVA with Tukey\strut
\end{minipage} & \begin{minipage}[t]{0.52\columnwidth}\raggedright
One quantitative (outcome) and one categorical (3-6 levels)\strut
\end{minipage}\tabularnewline
\begin{minipage}[t]{0.42\columnwidth}\raggedright
{[}3{]} Regression Model\strut
\end{minipage} & \begin{minipage}[t]{0.52\columnwidth}\raggedright
Same as either {[}1b{]} or {[}2{]}, plus one quantitative
(covariate)\strut
\end{minipage}\tabularnewline
\begin{minipage}[t]{0.42\columnwidth}\raggedright
{[}4{]} 2x2 Table\strut
\end{minipage} & \begin{minipage}[t]{0.52\columnwidth}\raggedright
Two categorical (2 level) variables\strut
\end{minipage}\tabularnewline
\begin{minipage}[t]{0.42\columnwidth}\raggedright
{[}5{]} JxK Table\strut
\end{minipage} & \begin{minipage}[t]{0.52\columnwidth}\raggedright
Two categorical variables, one with 2-6, other with 3-6 levels\strut
\end{minipage}\tabularnewline
\begin{minipage}[t]{0.42\columnwidth}\raggedright
{[}6{]} 2x2xJ Table\strut
\end{minipage} & \begin{minipage}[t]{0.52\columnwidth}\raggedright
Same as {[}4{]}, plus one categorical with 3-6 levels\strut
\end{minipage}\tabularnewline
\bottomrule
\end{longtable}

\hypertarget{analysis-1-comparing-the-means-of-two-populations}{%
\subsection{Analysis 1: Comparing the Means of Two
Populations}\label{analysis-1-comparing-the-means-of-two-populations}}

Here, you will choose either to use a paired samples design or an
independent samples design.

If you're using paired samples (to do Analysis 1a), then you will
specify

\begin{itemize}
\tightlist
\item
  Outcome A, a quantitative variable, and
\item
  Outcome B, also a quantitative variable.
\end{itemize}

If you're using \textbf{independent} samples (to do Analysis 1b), then
you will specify

\begin{itemize}
\tightlist
\item
  Outcome C, a quantitative variable, and
\item
  Factor Z, a two-level categorical variable
\end{itemize}

Each level of your Factor Z must apply to a minimum of 10 subjects in
the Survey.

\hypertarget{analysis-2-comparing-the-means-of-three-or-more-populations}{%
\subsection{Analysis 2: Comparing the Means of Three or More
Populations}\label{analysis-2-comparing-the-means-of-three-or-more-populations}}

Here, you will complete an analysis of variance, with pre-planned Tukey
HSD comparisons. You will specify:

\begin{itemize}
\tightlist
\item
  Outcome D, a quantitative variable (which can repeat A, B or C from
  before if you like)
\item
  Factor Y, a 3-6 level categorical variable
\end{itemize}

Each level of your Factor Y must apply to a minimum of 6 subjects in the
Survey.

\hypertarget{analysis-3-regression-model-with-one-covariate}{%
\subsection{Analysis 3: Regression Model with One
Covariate}\label{analysis-3-regression-model-with-one-covariate}}

Here, you will use the same outcome and factor as you used in either
Analysis 1 (if you used independent samples) or Analysis 2, but add a
new covariate. So you will specify:

\begin{itemize}
\tightlist
\item
  Outcome E, which must be the same as either your Outcome C (if you're
  amplifying Analysis 1b) or Outcome D (if you're amplifying Analysis
  2).
\item
  Factor X, which must be the same as either Factor Z (if you're
  amplifying Analysis 1b) or Factor Y (if you're amplifying Analysis 2),
  but now you're also adding:
\item
  Covariate G, which is to be a quantitative variable not used in
  Analyses 1 or 2.
\end{itemize}

\hypertarget{analysis-4-comparing-two-population-proportions}{%
\subsection{Analysis 4: Comparing Two Population
Proportions}\label{analysis-4-comparing-two-population-proportions}}

Here, you will develop and analyze a 2x2 contingency table. You will
specify:

\begin{itemize}
\tightlist
\item
  Factor L (which needs to have exactly 2 levels) and will be in the
  rows of your table, and
\item
  Factor M (which also needs to have exactly 2 levels) and be in the
  columns.
\end{itemize}

Every cell in your 2x2 table needs to have at least 5 observations. You
are welcome to re-use a two-level factor you've used in a previous
Analysis for L or M, but must add a new factor for the other.

\hypertarget{analysis-5-a-larger-two-way-table}{%
\subsection{Analysis 5: A Larger Two-Way
Table}\label{analysis-5-a-larger-two-way-table}}

Here, you will develop a contingency table analysis to describe (in the
rows) a factor with 2-6 levels, and (in the columns) a factor with 3-6
levels. You will specify:

\begin{itemize}
\tightlist
\item
  Factor J (which must have 2-6 levels) and will be in the rows of this
  table, and
\item
  Factor K (which must have 3-6 levels) and will be in the columns
\end{itemize}

You can re-use at most one of Factors L and M as Factor J, but Factor K
must be new.

\hypertarget{analysis-6-comparing-population-proportions-in-a-2x2xn-contingency-table}{%
\subsection{Analysis 6: Comparing Population Proportions in a 2x2xN
contingency
table}\label{analysis-6-comparing-population-proportions-in-a-2x2xn-contingency-table}}

Here, you will amplify Analysis 4 by developing a Mantel-Haenszel
analysis of a contingency table with 2 rows (re-using Factor L from
Analysis 4), 2 columns (re-using Factor M from Analysis 4) and 3-6
layers (or strata) in a new factor called Factor N. You wlil specify:

\begin{itemize}
\tightlist
\item
  Factor N (which must have 3-6 levels) and can repeat Factor J or K if
  you like, so long as it is different from Factors L or M.
\end{itemize}

\hypertarget{backups}{%
\subsection{Backups}\label{backups}}

You will also specify

\begin{itemize}
\tightlist
\item
  Quantitative Variable Q: a backup quantitative variable for use as an
  outcome
\item
  Quantitative Variable R: another backup quantitative variable, just in
  case
\item
  Factor Variable S: a backup 2 level categorical variable for use as a
  group
\item
  Factor Variable T: a backup 3-6 level categorical variable for use as
  a group, and
\end{itemize}

These backups are for whether the results of the Survey turn out to
yield either outcome variables with no variation at all across the
groups of interest (in Analyses 1-3) or tables with insufficiently
populated cells (in Analysis 4-6).

\hypertarget{table-of-what-youll-specify-on-the-form}{%
\section{Table of What You'll Specify on the
Form}\label{table-of-what-youll-specify-on-the-form}}

You will specify the following elements on your form (remember that you
will either specify A and B if you choose Analysis 1a, or C and Z if you
choose Analysis 1b.)

\begin{longtable}[]{@{}rcrr@{}}
\toprule
Analysis & Description & Item \# & Item Name\tabularnewline
\midrule
\endhead
1a & A (quantitative) & -- & --\tabularnewline
1a & B (quantitative) & -- & --\tabularnewline
1b & C (quantitative) & -- & --\tabularnewline
1b & Z (two-category) & -- & --\tabularnewline
2 & D (quantitative) & -- & --\tabularnewline
2 & Y (3-6 category) & -- & --\tabularnewline
3 & E (quantitative, same as C or D) & -- & --\tabularnewline
3 & X (same as either Y or Z) & -- & --\tabularnewline
4 \& 6 & L (two-category) & -- & --\tabularnewline
4 \& 6 & M (two-category) & -- & --\tabularnewline
5 & J (2-6 category) & -- & --\tabularnewline
5 & K (3-6 category) & -- & --\tabularnewline
6 & N (3-6 category) & -- & --\tabularnewline
Backup & Q (quantitative) & -- & --\tabularnewline
Backup & R (quantitative) & -- & --\tabularnewline
Backup & S (two-category) & -- & --\tabularnewline
Backup & T (3-6 category) & -- & --\tabularnewline
\bottomrule
\end{longtable}

\hypertarget{taskE}{%
\chapter{Task E (Taking the Survey) Instructions}\label{taskE}}

\hypertarget{deadline-and-submission-information-4}{%
\section{Deadline and Submission
information}\label{deadline-and-submission-information-4}}

Task E is due at noon on 2018-10-31. Submit your answers to the course
survey via the Google Form linked at \textbf{LINK GOES HERE}. That link
will go live after class on 2018-10-25.

\begin{itemize}
\tightlist
\item
  The final item asks for your name, and the system is collecting your
  email address (you must be logged into Google via CWRU). These will be
  pruned from the survey before data sets are created.

  \begin{itemize}
  \tightlist
  \item
    You should answer all of the items. Please don't skip any items you
    can answer. Your colleagues need data.
  \item
    If you want to save your work and return later, note that only the
    \emph{first} item in each section of the survey must be completed
    for Google to let you submit your work. Once you've submitted a
    partially completed survey, you can return as often as you like
    before the deadline to finish up.
  \end{itemize}
\end{itemize}

\hypertarget{receiving-your-study-1-data-early-november}{%
\section{Receiving Your Study 1 Data (early
November)}\label{receiving-your-study-1-data-early-november}}

Once the Survey is complete, we will post \textbf{two} data files for
you, each containing some of the variables you need.

\begin{itemize}
\tightlist
\item
  You will need to download both files, and then \emph{combine} and tidy
  to suit your needs. Combining two data sets like this is a skill
  you'll need to master to successfully complete the Project.
\item
  The two files will be linked by the subject \texttt{id} number.
\end{itemize}

\hypertarget{taskF}{%
\chapter{Task F (Sharing Study 2 Data) Instructions}\label{taskF}}

\hypertarget{deadline-and-submission-information-5}{%
\section{Deadline and Submission
information}\label{deadline-and-submission-information-5}}

Task F is due at noon on 2018-11-14. Submit your Task F work through
\href{https://canvas.case.edu/}{Canvas}.

\hypertarget{sharing-your-data-appropriately}{%
\section{Sharing Your Data
Appropriately}\label{sharing-your-data-appropriately}}

Task F requires you to share your data for Study 2. The model for this
Task is Jeff Leek's \href{https://github.com/jtleek/datasharing}{Guide
to Data Sharing}, which you should definitely read.

Specifically, you will submit the following via Canvas on time.

\begin{enumerate}
\def\labelenumi{\arabic{enumi}.}
\tightlist
\item
  a direct link to the raw data set (without any need for me to sign up
  for anything) or a .csv copy of the raw data set called
  \texttt{yourname-raw.csv}
\item
  a single .csv file with a name of your choice containing a clean, tidy
  data set for Study 2, along with
\item
  a Word or PDF file containing both

  \begin{enumerate}
  \def\labelenumii{\alph{enumii}.}
  \tightlist
  \item
    a \textbf{codebook} section which describes every variable (column)
    and its values in your .csv file,
  \item
    a \textbf{study design} section which reminds (and updates) us about
    the source of the data and your research question.
  \end{enumerate}
\end{enumerate}

If you must zip the raw data set or tidy data set, OK, but include the
Word or PDF file as a separate document, apart from the zip file you
submit. The details for these elements of Task F follow.

\hypertarget{the-raw-data-set}{%
\section{The Raw Data Set}\label{the-raw-data-set}}

You need to show me the raw, de-identified data. The data are raw if
you:

\begin{itemize}
\tightlist
\item
  Ran no software on the data and Did not manipulate any of the numbers
  in the data
\item
  You did not remove any data from the data set other than to
  de-identify it and eliminate protected information and anything else
  that you cannot share
\item
  You did not summarize the data in any way
\end{itemize}

A direct link (without me having to sign up for anything) is preferred.
If this is not possible, send a .csv file of the raw data set, called
\texttt{yourname-raw.csv}. Note that you should not send me any
variables you have no chance of using in your analyses, but may include
some variables you haven't made a final decision on.

\hypertarget{the-tidy-data-set}{%
\section{The Tidy Data Set}\label{the-tidy-data-set}}

Your .csv file should include only those variables you will actually use
in your analysis of Study 2. Your .csv file should include one row per
subject in your data, and one column for each variable you will use.
Your data are tidy if each variable you measure is in its own column,
and each different observation of that variable is in its own row,
identifed by the subject identification code in \texttt{Subj\_ID}.

You need to provide:

\begin{enumerate}
\def\labelenumi{\arabic{enumi}.}
\tightlist
\item
  a header row (row 1 in the spreadsheet) that contains full row names.
  So if you measured age at diagnosis for patients, you would head that
  column with the name \texttt{AgeAtDiagnosis} or
  \texttt{Age.at.Diagnosis} instead of something like \texttt{ADx} or
  another abbreviation that may be hard for another person (or you, two
  years from now) to understand.
\item
  a study identification number (I would call this variable
  \texttt{Subj\_ID} and use consecutive integers to represent the rows
  in your data set) which should be the left-most variable in your tidy
  data.
\item
  a quantitative outcome with a meaningful name using no special
  characters other than a period (\texttt{.}), hyphen(\texttt{-}) or
  underscore (\texttt{\_}) used to separate words, which should be the
  second variable in your data.

  \begin{itemize}
  \tightlist
  \item
    If you have any missing \textbf{outcome} values, \textbf{delete
    those rows} entirely from your tidy data set.
  \end{itemize}
\item
  at least four predictor variables, each with a meaningful name using
  no special characters other than \texttt{.} or \texttt{\_} to separate
  words, and the predictors should be shown in columns to the right of
  the outcome.

  \begin{itemize}
  \tightlist
  \item
    \emph{Continuous} variables are anything measured on a quantitative
    scale that could be any fractional number.
  \item
    \emph{Ordinal categorical} data are data that have a fixed, small
    (\textless{} 100) number of levels but are ordered.
  \item
    \emph{Nominal categorical} data are data where there are multiple
    categories, but they aren't ordered.
  \item
    Categorical predictors should read into R as factors, so your
    categories should include letters, and not just numbers. In general,
    try to avoid coding nominal or ordinal categorical variables as
    numbers.
  \item
    Label your categorical predictors in the way you plan to use them in
    your analyses
  \item
    \emph{Missing data} are data that are missing and you don't know the
    mechanism. Missing data in the predictor variables are allowed, and
    you should code missing values in your tidy data set as \texttt{NA}.
    It is critical to report if there is a reason you know about that
    some of the data are missing. You should also not impute/make
    up/throw away missing observations on the predictor values in your
    tidy data set that you submit for Task F.
  \end{itemize}
\item
  any other variables you need to share with me (typically this would
  only include things you had to use in order to get to your final
  choice of outcome and predictors.) Most people will not need to share
  any additional variables.
\end{enumerate}

I will need to be able to take your submitted \texttt{.csv} file and run
your eventual Markdown file (part of Task H) against it and obtain your
results, so it must be completely clean. Because it is a \texttt{.csv}
file, you'll have no highlighting or bolding or any other special
formatting. If you have missing values, they should be indicated as NA
in the file. If you obtain the file in R, and then write it to a .csv
file, you should write the file without row numbers if you already have
an identification variable. To do so, you should be able to use
\texttt{write.csv(dataframeinR,\ "newfilename.csv",\ row.names\ =\ FALSE)}
where you will substitute in the name of your data frame in R, and new
(.csv) file name. Don't use the same name for your original data set and
your tidy one.

\hypertarget{the-codebook}{%
\section{The Codebook}\label{the-codebook}}

For almost any data set, the measurements you calculate will need to be
described in more detail than you will sneak into the spreadsheet. The
code book contains this information. At minimum it should contain:

\begin{enumerate}
\def\labelenumi{\arabic{enumi}.}
\tightlist
\item
  Information about the variables (including units! and codes for any
  categorical variables) in your tidy data set
\item
  Information about the summary choices or transformations you made or
  the development of any scales from raw data
\end{enumerate}

By reading the codebook, I should understand what you did to get from
the raw data to your tidy data, so add any additional information you
need to provide to make that clear.

\hypertarget{the-study-design}{%
\section{The Study Design}\label{the-study-design}}

Here is where I want you to put the information about the experimental
study design you used. You can and should reuse (and edit) the
information you provided as part of the Proposal in this Codebook. The
material you need here consists of three parts from the proposal,
updated to mirror your current plan. Specifically, you should provide:

\begin{enumerate}
\def\labelenumi{\arabic{enumi}.}
\tightlist
\item
  Your research question describes your outcome, your key predictor and
  other predictors, and the population of interest. It is probably
  easiest to follow one of these formats\footnote{You are welcome to
    move the clauses around to make for a clearer question.}.
\end{enumerate}

\begin{itemize}
\tightlist
\item
  What is the effect of \texttt{*your\ key\ predictor*} on
  \texttt{*your\ outcome*} adjusting for
  \texttt{*your\ list\ of\ other\ predictors*} in
  \texttt{*your\ population\ of\ subjects*}?
\item
  How effectively can \texttt{*specify\ your\ predictors*} predict
  \texttt{*your\ outcome*} in \texttt{*your\ population\ of\ subjects*}?
  or
\end{itemize}

\begin{enumerate}
\def\labelenumi{\arabic{enumi}.}
\setcounter{enumi}{1}
\item
  A thorough description of the data collection process, with complete
  details as to the nature of the variables, the setting for data
  collection, and complete details of any apparatus you used which may
  affect results that \textbf{has not already been covered} in the
  codebook materials.
\item
  Specification of the subjects and methods involved.

  \begin{enumerate}
  \def\labelenumii{\alph{enumii}.}
  \tightlist
  \item
    Who are the subjects under study? How many are included in your
    final tidy data set?
  \item
    When were the data gathered? By whom?
  \item
    What caused subjects to be included or excluded from the study?
  \end{enumerate}
\end{enumerate}

\hypertarget{taskG}{%
\chapter{Task G (The Project Update) Instructions}\label{taskG}}

\hypertarget{deadline-and-submission-information-6}{%
\section{Deadline and Submission
information}\label{deadline-and-submission-information-6}}

Task G is due at noon on 2018-11-28. Submit your Task G work through
\href{https://canvas.case.edu/}{Canvas}.

\hypertarget{taskH}{%
\chapter{Task H (The Portfolio) Instructions}\label{taskH}}

Task H requires you to provide a written portfolio of materials, which
you will also make use of in your final presentation.

Details to come.

\hypertarget{taskI}{%
\chapter{Task I (Your Presentation) Instructions}\label{taskI}}

Details to come.

\bibliography{book.bib,packages.bib}


\end{document}
